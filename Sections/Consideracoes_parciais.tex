\label{chapter:consideracoes}


\par 
\textcolor{red}{A mineração de dados se tornou uma ferramenta de suporte com papel essencial no gerenciamento da informação dentro das organizações. O manuseamento dos dados e a análise das informações que eram feitas de maneira convencional, se tornou impossível devido a imensa quantidade de dados que é coletado e armazenado em sua base diariamente. Encontrar padrões escondidos e relacionamentos em arquivos que possuem um grande volume de informações de forma manual, deixou de ser uma escolha. As técnicas de mineração passaram a ser utilizadas e começaram a estar presentes no nosso cotidiano.}

\par
\textcolor{red}{Com uma grande quantidade de dados que estão sendo armazenados pelas universidades e com recente surgimento de várias tecnologias e técnicas de extração de dados, uma boa parte dessas instituições estão tentando de alguma forma utilizar esses dados que são extraídos  na intenção de compreende-los e utiliza-los para benefício próprio. Como podemos observar nos trabalhos apresentado neste TCC, que através dos dados armazenado de candidatos que vão prestar o vestibular, foi gerado um perfil dos mesmos na intenção de descobrir os motivos que influenciam os seus desempenhos na nota da prova que eles fizeram.}

\par
\textcolor{red}{Assim como os trabalhos apresentados, este trabalho possui o mesmo objetivo que os deles, que é a de gerar um perfil dos candidatos para saber os fatores que influenciam o seu desempenho na pontuação da prova, com intuito de que com os resultados obtidos, de alguma forma, possa ofereça como apoio para que o MEC analise a sua estratégia no plano de ensino das escolas e também para que o MDS com o conhecimento obtido desses resultados, possa solucionar os problemas socioeconômico dos candidatos que foram encontrado. Se espera também, que com o perfil dos candidatos que foram gerados, a UFRR tome futuras ações com essas informações para benefício próprio.}

\textcolor{red}{Em resumo, podemos perceber que a mineração de dados vem  cada vez mais sendo utilizada dentro da área educacional como em universidades, pois, com a imensa quantidade de dados que é armazenado, onde muitas vezes é repleta de informações importantes escondidas, se necessita dessas técnicas de mineração para se extrair esses dados com a intenção de usa-las, podendo assim relacionar informações contidas para predeterminar possíveis ações futuras na sua gestão com tutores, alunos e entre outros. Não resta dúvida de que a mineração de dados na área educacional está sendo extremamente promissora e que, apesar dos resultados já alcançados, ainda ela tem muito para o que oferecer.}

