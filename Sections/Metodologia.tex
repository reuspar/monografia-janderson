\label{chapter:metodo}

\par
\textcolor{red}{Com a grande quantidade de dados que são armazenados dos seus acadêmicos, as faculdades muitas vezes não sabe em como utilizá-los para benefício próprio. Com base nesses dados armazendos, especificamente os dados do candidatos que prestaram o vestibular, se pretende aplicar a mineração de dados educacional, com o objetivo de extrair o conhecimento dessa base para saber quais fatores que afetam o desempenho desse candidato em sua nota do vestibular. Neste contexto o trabalho proposto segue as seguintes etapas para se alcançar esse objetivo.}

\par
\textcolor{red}{Para a fase de pré-processamento será analisado o banco de dados do vestibular da Universidade Federal de Roraima (UFRR) do ano de 2017 fornecido pela Comissão Permanente de Vestibular (CPV), que esta encarregada de gerenciar os dados geral dos candidatos inscritos. Para formar a base de dados onde será aplicado a mineração, estará sendo feita a junção dos dados dos questionarios socioeconômico que contém no total de 37 questões e os dados de cadastro geral que são realizados pelos candidatos durante a etapa de inscrição.}


\par
\textcolor{red}{Como em todo banco de dados é normal encontrar inconsistência de dados, como ruídos, itens de dados em branco, valores absurdos e erros de digitação, para grantir a qualidade dos dados, serão necessário corrigir esses problemas para que na hora da mineração não ocorra nenhum erro ou que tenha algum resultado incoerente. Tambem será necessario ser feito a eliminação de atributos repetidos assim como dados inrrelevantes para a pesquisa e dos dados de candidatos que não compareceram para prestar o vestibular.}

\par
\textcolor{red}{Depois de selecionar e transformar a base de dados em um formato apropriado, será aplicado a técnica de mineração de ávore de decisão para a análise de dados dos candidatos, a escolha desse método foi pelo seguinte motivo, que segundo \citeonline{Simon2017}, esta técnica serve como um sistema de suporte a decisão, que é utilizado para aprender funções de classificação que mostra o valor de uma variável dependente por meio de dados de variáveis independentes fornecidos. Para alguns autores a técnica de árvore de decisão é considerada uma das abordagens mais poderasas em mineração de dados e descoberta do conhecimento.}

\par
\textcolor{red}{Dentre os algoritmos de árvore de decisão, foi optado em utilizar o algoritmo J48, pois, segundo \citeonline{Simon2017} e \citeonline{Martinhago2005} que utilizaram o mesmo algoritmo, o algoritmo J48 gera a árvore de decisão e a tranforma em regras de classidicação, além de que na literatura é bastante utilizado em aplicações educacionais. O software utilizado para a tarefa de classificação e mineração dos dados é o WEKA que já tem o algoritmo J48 disponível nele, que segundo \citeonline{Simon2017} o WEKA é utilizado a muito tempo em pesquisas que envolvem mineração de dados, sendo aceito amplamente tanto em empresas como academias tendo uma comunidade muito grande que ainda esta ativa.}

\par
\textcolor{red}{A ferramenta WEKA foi escolhida pelo motivo de possuir todos os classificadores implementados, além de fornecer outras funcionalidades para pré-processamento, classificação, regressão, agrupamentos, regras de associação, visualização e entre outros \cite{Camilo2009}. Segundo \citeonline{Amaral2016} algumas vantagens de se utilizar o software WEKA é que ela é uma ferramenta gratuita, madura que esta disponivel desde os anos 90 com inúmeros algoritmos que executam diversas tarefas de aprendizagem de maquina e possui uma interface gráfica interativa, facil de manusear sem a necessidade de se digitar códigos.} 

\par
\textcolor{red}{Para consolidar os dados obtidos pelo algoritmo de árvore de decisão, será aplicado a técnica de associação de dados utilizando o algoritmo apriori, segundo \citeonline{Librelotto2014} ele pode executar varias passagens pelo banco além de trabalhar com um número grande de atributos, tendo como resultado varias alternativas combinadas entre eles atravé das buscas sucessivas em toda a base de dados. O motivo de se ter essas buscas sucessivas é que ele usa o mesmo raciocínio de dividir para conquistar, com intuito de encontrar regras de associação para todos os atributos possíveis, executando um processo de indução de regras para todas as combinações possíveis de atributos.}

\par
\textcolor{red}{Para avaliar os resultados obtidos depois de todos os testes, será utilizado a métrica de avaliação de modelo matriz de confusão para determinar se o resultado previsto comrresponde ao resultado real. Ao final dessas etapas, a partir dos resultados e conhecimento extraído, se espera gerar um perfil para poder encontrar os principais fatores que influencia o bom ou o mau desempenho das notas dos candidatos para o vestibular. Através da associação de dados, com as regras geradas se pode ter como conhecimento relevante para futuras ações da UFRR, referente ao perfil dos candidatos que prestaram seletivo para o vestibular.}

\par 
\textcolor{red}{Como este trabalho tem como intuito uma análise mais profunda dos fatores que afetam o desempenho dos vestibulandos, para poder observar os padrões que tem mais em comum entre eles, pode-se considerar que esta análise esta voltada para o desenvolvimento social dos candidatos. Com isso, com os resultados obtidos, pode servir como contribuição para o MEC avaliar em como esta indo a sua estratégia do plano de ensino para as escolar publicas, federais e privadas e para o Ministério de Desenvolvimento Social (MDS) para tentar amenizar os possiveis problemas socioeconômico encontrados.}

