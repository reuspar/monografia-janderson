%===========================================================
%INTRODUÇÃO
%===========================================================

\par
Devido ao sucessivo avanço da tecnologia em nossa sociedade, o número de dados gerados nesses ultimos anos cresceu de uma forma explosiva. Segundo \citeonline{Observatory2013}, de 2012 à 2014 foram gerados cerca de 90\% dos dados, sendo que tem como fontes documentos de textos, transações comerciais, streaming de áudios e vídeos, e-mails, telecomunicações, dados pessoais e entre diversas outras. Juntando este crescimento excessivo na quantidade imensa de informações geradas com a grande evolução que é realizado na área de armazenamento de dados, o resultado obtido é uma verdadeira mina de ouro escondida em terabytes ou até mesmo em petabytes de dados \cite{Carvalho2014}. 

\par
Diante desse cenário com uma grande quantidade de informações, surge a necessidade de se utilizar técnicas e ferramentas computacionais para administrar, gerenciar e analisar tais informações. No mundo em que vivemos, atualmente vem crescendo a participação dos computadores na sociedade em inúmeros ramos de atividades como social, ciêntifica, saúde e econômica, sendo que já existe computadores que armazenam o que foi efetuado, medido, calculado e decidido desses serviços. Contudo, muito dessas decisões que são tomadas, não se tem o conhecimento suficinte das informações que provém dos dados acumulados em bases de dados de sistemas transacionais \cite{Rabelo2007}.

\par
Com base nisso, para atender esse contexto, surgiu uma nova área conhecida como Descoberta do Conhecimento em Banco de Dados (KDD), uma área da computação que tem como intuito de buscar e extrair informações úteis a respeito dos dados. Em varios casos, esse processo pode ser longo e trabalhoso \cite{Stulp2014}. O KDD é descrita como um processo bastante complexo onde tem como objetivo de extrair o conhecimento gerado de grande quantidade de dados, além de que, na sua estrutua é composto por três etapas principais: pré-processamento, mineração de dados e pós-processamento \cite{Rabelo2007}.

\par
Para muitos, uma das etapas mais importante e complexa de todo o método do procedimento de descoberta do conhecimento é a etapa de mineração de dados (MD). O MD utiliza de ferramentas e técnicas no objetivo de procurar encontrar dados importantes que estão ocultos em uma enorme quantidade de dados absorvidos por um banco de dados para poder dar suporte à tomada de decisão. O processo de mineração de dados segue as seguintes etapas: o reconhecimento do problema, o pré-processamento dos dados, a extração de padrões dos dados obtidos, a definição de uma tarefa e a escolha do algoritmo adequado para a mineração e o pós-processamento que inclui a análise dos resultados obtidos \cite{Stulp2014}. 

\par
A mineração de dados possui varias áreas onde ela é aplicada, uma delas é a mineração de dados educacionais (MDE). O MDE é um campo em desenvolvimento que consiste em examinar uma grande quantidade de dados educacionais com o intuito de obter padrões de dados não triviais. Esse campo utiliza técnicas que verificam padrões em contexto educacional para poder criar parâmetros cognitivos de grande complexidade que proporcione de entender o processo de aprendizado, como por exemplo, na utilização de sistemas que recomendação para aumentar o aprendizado de um determinado estudante durante o seu processo de estudo, no caso, informando em quais conteúdos eles precisam melhorar.

\par
\textcolor{red}{Dessa forma, dentro deste contexto, esta pesquisa, que tem como caráter exploratorio, visa em criar um modelo preditivo através das técnicas de mineração de dados, Árvore de decição e Associação de Dados, que será aplicada nos dados do questionario socioeconômico dos candidatos que prestaram ao vestibular da Universidade Federal de Roraima no ano de 2017, dados esses que foram fornecidos pelo Comissão Permanente de Vestibular, que tem como intuito de gerar regras para poder criar um perfil e identificar os fatores que causaram o mal desempenho do candidato no resultado da sua prova.}







% %===========================================================
% %MOTIVAÇÃO
% %===========================================================
\section{Motivação}

\textcolor{red}{Recentemente, as faculdades tem cada vez mais armazenado dados sobre os alunos e candidatos que participaram do vestibular, isso se deve, ao  grande aumento da redução dos custos e facilidade para manter essas informações. Este aumento da quantidade de dados, no entanto, dificulta uma pesquisa mais precisa, fazendo com que ocorra o risco de o mesmo se transformar em apenas um acúmulo de informações sem utilidade. }

\par
\textcolor{red}{Nesse contexto, o uso da Descoberta de Conhecimento em Banco de Dados (KDD) vem ganhando interesse e importância por parte dos donos dessa grande quantidade de dados, pois, as pesquisas obtidas nessa área tem em vista à descoberta de técnicas e tecnologias mais eficientes para a recuperação de dados, procurando encontrar conhecimentos escondido que possam ser úteis, como por exemplo, para as faculdades que podem conhecer melhor os seus alunos atráves dos dados obtido.}

\textcolor{red}{Então, para que se torne possível compreender esses dados que são armazendos, no objetivo de solucionar o problema proposto e de se obter um resultado relevante, este trabalho visa em analisar algumas técnicas de mineração proposta na literatura e selecionar a mais adequada no objetivo de se criar um modelo preditivo para gerar regras que indicará os fatores que influênciam tais alunos em suas notas do vestibular.}


%===========================================================
%DEFINIÇÃO DO PROBLEMA
%===========================================================
\section{Definição do problema}

\par
\textcolor{red}{Com uma quantidade enorme de dados armazenado dos alunos pelas faculdade, muitas não sabem de como utiliza-las para o seu aproveito, consequentemente, muitos desses dados acabam ficando acumulados sem nenhuma utilidade. Contudo, muitos desses dados tem uma grande importância para se obter informações preciosas dos candidatos, sendo que se pode identificar problemas para poder tentar relsover alguns deles. Diante disso, como não existe um padrão consolidado para minerar esses dados, este trabalho visa analisar as técnicas de mineração de dados propostas e selecionar a mais adequada para minerar a base de dados da Comissão Permanente de Vestibular, com intuito de coompreender o perfil dos candidatos analisando os fatores que influênciam no seu desempenho no vestibular.}

\par
\textcolor{red}{O problema deste trabalho pode então, ser expresso na seguinte questão: \textbf{Como analisar o perfil dos vestibulandos, aplicando técnicas de mineração na base de dados que contém fatores socioeconômicos e cadastro geral do candidato, no sentido de criar um modelo preditivo de tal forma que gere regras através dos dados fornecidos, para poder saber quais fatores que influênciam no seu desempenho na nota  da prova do vestibular? }}

%O problema considerado neste trabalho é expresso na seguinte questão: \textbf{Como complementar e aprimorar a verificação de propriedades de segurança em circuitos lógicos, de tal forma que uma propriedade possa ser mapeada em um problema de alcançabilidade simbólica, por exemplo, se é possível alcançar um estado específico (para uma dada propriedade) a partir do estado inicial?}


%===========================================================
%OBJETIVOS GERAIS E ESPECIFICOS
%===========================================================

\section{Objetivo Geral}

\par
\textcolor{red}{O objetivo principal deste trabalho é de eleger uma técnica de mineração de dados, a partir das técnicas já proposta na literatura, e aplica-la na base de dados do vestibular da UFRR, visando em criar um modelo preditivo que prevê quais os fatores que contribue ou afeta o desempenho na nota da prova do candidato no vestibular.}



\section{Objetivos Específicos}

Para alcançar o objetivo geral, os seguintes objetivos específicos necessitam ser executados:


\begin{enumerate}
  \item Propor um método para especificar pré- e pós-condições de circuitos digitais em nível de portas lógicas descritos em VHDL;
  \item Analisar ferramentas de verificação de modelos que utilização a técnica \textit{Bounded Model Checking};
  \item Especificar uma técnica de conversão de circuitos em linguagem de descrição de hardware VHDL para um modelo a ser verificado usando a técnica \textit{Bounded Model Checking};
   \item \textcolor{red}{Desenvolver} um método para identificação de estados de erros ou ocorrências indevidas, de um dado circuito analisado, em modelos de hardware descritos em nível de bit na linguagem de descrição VHDL.
  \item Validar a aplicação do método proposto sobre \textit{benchmarks} públicos de programas em VHDL, a fim de examinar a sua eficácia e aplicabilidade.
\end{enumerate}

%===========================================================
%METODOLOGIA PROPOSTA
%===========================================================
\section{Metodologia proposta}

Esta seção descreve as principais etapas que foram identificadas para alcançar os objetivos \textcolor{red}{deste trabalho}. Estas etapas fornecem os passos necessários e direções para desenvolver a metodologia proposta e podem ser descrita em três diferentes fases como segue: análise do domínio, metodologia proposta e validação da metodologia.

Na etapa de análise de domínio, toda a teoria necessária para entender os métodos, técnicas e ferramentas aplicadas à metodologia de desenvolvimento/validação de circuitos lógicos utilizando a linguagem de descrição de hardware VHDL serão analisadas e avaliadas. Na fase da metodologia proposta, uma versão inicial da metodologia para verificação de circuitos lógicos, tem como foco uma ou mais restrições, e seu escopo serão precisamente definidos e propostos. Depois disso, esta metodologia é mais adiante refinada na fase de validação aplicando-a na verificação dos estudos de caso.

Com o intuito de realizar as atividades deste trabalho, uma abordagem iterativa e incremental será usada com o propósito de reduzir riscos e incertezas. Sendo assim, para cada incremento da solução proposta, as três etapas nomeadas nesta proposta como análise do domínio, metodologia proposta e validação da metodologia podem ser tratadas com diferentes ênfases em cada fase do trabalho.

Por exemplo, no início deste trabalho, a análise de domínio provavelmente terá maior ênfase do que as outras fases metodologia proposta e validação da metodologia. Na metade do projeto, a fase de metodologia proposta provavelmente terá mais ênfase do que as outras duas fases. Finalmente, a fase de validação da metodologia provavelmente terá mais ênfase no fim do desenvolvimento deste TCC. A principal razão para adotar uma abordagem iterativa e incremental é desenvolver o TCC incrementalmente, permitindo assim tirar vantagem do que foi aprendido durante cada incremento do projeto.

% Para cada incremento do TCC, relatórios técnicos devem ser escritos com o propósito de descrever os principais feitos alcançados em uma dada iteração. Além disso, se os resultados significantes têm sido alcançados, então artigos científicos podem ser escritos para reportá-los à comunidade acadêmica através das publicações em workshops e conferências nacionais/internacionais. Potencialmente, os artigos científicos podem ser produzidos em cada incremento do projeto com o intuito de fornecer claramente o progresso deste projeto de pesquisa.

%===========================================================
%CONTRIBUIÇÕES PROPOSTAS
%===========================================================
\section{Contribuições propostas}

As contribuições propostas para este trabalho são:
\begin{itemize}
  \item O desenvolvimeno de um método para verificação de hardware com o intuito de facilitar os passos da verificação e ao mesmo tempo reduzir substancialmente o tempo de verificação de projetos de hardware descritos em VHDL;
  \item Este trabalho apresenta para o método proposto, o desenvolvimento e implementação de uma ferramenta de verificação de circuitos lógicos em VHDL com a integração da ferramenta ESBMC (\textit{Efficient SMT-Based Context-Bounded Model Checker})\cite{cordeiro2012smt} na análise.
\end{itemize}


%===========================================================
%ORGANIZAÇÃO DO TRABALHO
%===========================================================
\section{Organização do trabalho}
A introdução deste trabalho apresentou: o contexto, definição do problema, objetivos, metodologia e contribuições dessa pesquisa. Os capítulos restantes são organizados da seguinte forma:

\par
No \autoref{chapter:conceitos} \textbf{Conceitos e Definições}, são apresentados os conceitos abordados neste trabalho, especificamente: Linguagens de descrição de hardware, Verificação e Validação de Sistemas e Técnicas de Compiladores.

\par
No \autoref{chapter:correlatos} \textbf{Trabalhos Correlatos}, será apresentado o método de pesquisa utilizado, mas também, os resultados alcançados com a pesquisa, análise dos artigos e a contribuição dos mesmos para o desenvolvimento do método apresentado neste trabalho.

\par
No \autoref{chapter:metodo} \textbf{Método Proposto}, é descrito as etapas de execução do novo método proposto. Em especial, são descritos o método de transformação do código, a utilização das assertivas e a integração da ferramenta ESBMC no contexto do método.

\par
No \autoref{chapter:resultados} \textbf{Resultados Experimentais}, descreve-se a execução de uma avaliação experimental sobre o método proposto, bem como, 
\textit{benchmarks} utilizados para testes da ferramenta e os resultados obtidos através destes testes.
\par
E por fim no \autoref{chapter:consideracoes} \textbf{Considerações parciais e trabalhos futuros}, apresenta-se as considerações parciais e os trabalhos futuros a serem desenvolvidos. 
