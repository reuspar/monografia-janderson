\title{Monografia - Janderson}

\documentclass[
	% -- opções da classe memoir --
	12pt,				% tamanho da fonte
    oneside,			% Tipo de impressão, frente-verso(twoside) ou apenas frente(oneside)
	a4paper,			% tamanho do papel. 
	% -- opções da classe abntex2 --
	chapter=TITLE,		% títulos de capítulos convertidos em letras maiúsculas
	% -- opções do pacote babel --    
	english,			% idioma adicional para hifenização
	brazil				% o último idioma é o principal do documento    
	]{abntex2}
    
    %\renewcommand{\ABNTEXpartfontsize}{\normalsize}
	%\renewcommand{\ABNTEXchapterfontsize}{ \large}
	%\renewcommand{\ABNTEXsectionfontsize}{\normalsize}
	%\renewcommand{\ABNTEXsubsectionfontsize}{\normalsize}

% ---
% Pacotes básicos 
% ---
\usepackage{lmodern}			% Usa a fonte Latin Modern			
\usepackage[T1]{fontenc}		% Selecao de codigos de fonte.
\usepackage[utf8]{inputenc}		% Codificacao do documento (conversão automática dos acentos)
\usepackage{lastpage}			% Usado pela Ficha catalográfica
\usepackage{indentfirst}		% Indenta o primeiro parágrafo de cada seção.
\usepackage{color}				% Controle das cores
\usepackage{graphicx}			% Inclusão de gráficos
\usepackage{microtype} 			% para melhorias de justificação
\usepackage{todonotes}
\usepackage{tabularx}


\definecolor{algColor}{RGB}{255,206,206} % rgb(255, 206, 206)

% Pacotes para algoritmos/pseudo-código
\usepackage{listings}

\renewcommand{\lstlistingname}{Programa}

\lstset
{ %Formatting for code in appendix
    language=C,
    basicstyle=\footnotesize,
    numbers=left,
    stepnumber=1,
    frame = single,
    showstringspaces=false,
    tabsize=2,
    breaklines=true,
    xleftmargin=10pt,
    breakatwhitespace=false,
    extendedchars=true,
    literate={á}{{\'a}}1 
             {ã}{{\~a}}1
             {é}{{\'e}}1
             {í}{{\'i}}1
             {õ}{{\~o}}1
             {ç}{{\c{c}}}1,
}

\renewcommand{\lstlistlistingname}{Lista de códigos}

% Configura a ``Lista de Códigos'' conforme as regras da ABNT (para abnTeX2)
\begingroup\makeatletter
\let\newcounter\@gobble\let\setcounter\@gobbletwo
  \globaldefs\@ne \let\c@loldepth\@ne
  \newlistof{listings}{lol}{\lstlistlistingname}
  \newlistentry{lstlisting}{lol}{0}
\endgroup

\renewcommand{\cftlstlistingaftersnum}{\hfill--\hfill}

\let\oldlstlistoflistings\lstlistoflistings
\renewcommand{\lstlistoflistings}{%
   \begingroup%
   \let\oldnumberline\numberline%
   \renewcommand{\numberline}{\lstlistingname\space\oldnumberline}%
   \oldlstlistoflistings%
   \endgroup}
% ---
% Pacotes adicionais, usados apenas no âmbito do Modelo Canônico do abnteX2
% ---
\usepackage{lipsum}				% para geração de dummy text
% ---


% ---
% Pacotes de citações
% ---
%\usepackage[brazilian,hyperpageref]{backref}	 % Paginas com as citações na bibl
\usepackage[alf, abnt-etal-cite=2]{abntex2cite}	% Citações padrão ABNT

% --- 
% CONFIGURAÇÕES DE PACOTES
% --- 

\usepackage{abntex_ufrr_dcc}


% PACOTES DE ALGORITMO

\usepackage{algpseudocode,algorithm}
% Declaracoes em Português

\algrenewcommand\algorithmicend{\textbf{FIM}}
\algrenewcommand\algorithmicdo{\textbf{FAÇA}}
\algrenewcommand\algorithmicwhile{\textbf{ENQUANTO}}
\algrenewcommand\algorithmicfor{\textbf{PARA}}
\algrenewcommand\algorithmicforall{\textbf{PARA TODO}}
\algrenewcommand\algorithmicif{\textbf{SE}}
\algrenewcommand\algorithmicthen{\textbf{ENTÃO}}
\algrenewcommand\algorithmicelse{\textbf{SENÃO}}
\algrenewcommand\algorithmicreturn{\textbf{RETORNE}}
\algrenewcommand\algorithmicfunction{\textbf{FUNÇÃO}}
% New definitions
\algnewcommand\algorithmicswitch{\textbf{ESCOLHA}}
\algnewcommand\algorithmiccase{\textbf{CASO}}
\algnewcommand\algorithmicassert{\texttt{assert}}
\algnewcommand\Assert[1]{\State \algorithmicassert(#1)}%
% New "environments"
\algdef{SE}[SWITCH]{Switch}{EndSwitch}[1]{\algorithmicswitch\ #1\ \algorithmicdo}{\algorithmicend\ \algorithmicswitch}%
\algdef{SE}[CASE]{Case}{EndCase}[1]{\algorithmiccase\ #1}{\algorithmicend\ \algorithmiccase}%
\algtext*{EndSwitch}%
\algtext*{EndCase}%


% Rearranja os finais de cada estrutura
\algrenewtext{EndWhile}{\algorithmicend\ \algorithmicwhile}
\algrenewtext{EndFor}{\algorithmicend\ \algorithmicfor}
\algrenewtext{EndIf}{\algorithmicend\ \algorithmicif}
\algrenewtext{EndFunction}{\algorithmicend\ \algorithmicfunction}

% O comando For, a seguir, retorna 'para #1 -- #2 até #3 faça'
\algnewcommand\algorithmicto{\textbf{até}}
\algrenewtext{For}[3]%
{\algorithmicfor\ #1 $\gets$ #2 \algorithmicto\ #3 \algorithmicdo}


% ---
% Configurações do pacote backref
% Usado sem a opção hyperpageref de backref
%\renewcommand{\backrefpagesname}{Citado na(s) página(s):~}
% Texto padrão antes do número das páginas
%\renewcommand{\backref}{}
% Define os textos da citação
%\renewcommand*{\backrefalt}[4]{
%	\ifcase #1 %
%		Nenhuma citação no texto.%
%	\or
%		Citado na página #2.%
%	\else
%		Citado #1 vezes nas páginas #2.%
%	\fi}%
% ---

% ---
% Informações de dados para CAPA e FOLHA DE ROSTO
% ---
\titulo{Aplicação da técnica de ávore de decisão para avaliar os dados socioeconômicos dos candidatos do vestibular da UFRR de 2018 para prever o seu desempenho}

\autor{JANDERSON ALMEIDA DOS SANTOS}
\local{Boa Vista - RR}
\data{2018}
\orientador{Delfa Mercedes Huatuco Zuasnabar}

\tipotrabalho{Monografia}

\preambulo{Monografia de Graduação apresentada ao Departamento de Ciência da Computação da Universidade Federal de Roraima como requisito parcial para a obtenção do grau de Bacharel em Ciência da Computação.}

%Trabalho de conclusão de curso  na área de Verificação de Software desenvolvido na UFRR com o %objetivo \textcolor{red}{VERIFICAR O PADRÃO DOS OUTROS TCC} de gerar casos de teste para %sistemas embarcados críticos	

% ---

%-- Informações de dado para a FOLHA DE APROVAÇÃO
\renewcommand{\dataDefesa}{01 de Fevereiro de 2017}
\renewcommand{\orientadorBanca}{Prof. MSc.Delfa Mercedes Huatuco Zuasnabar}
\renewcommand{\primeiroMembroBanca}{Prof. MSc. Miguel Raymundo Flores Santibañez}
\renewcommand{\segundoMembroBanca}{Prof. MSc. Acauan Cardoso Ribeiro}

% ---
% Configurações de aparência do PDF final

% alterando o aspecto da cor azul
\definecolor{blue}{RGB}{41,5,195}

% informações do PDF
\makeatletter
\hypersetup{
     	%pagebackref=true,
		pdftitle={\@title}, 
		pdfauthor={\@author},
    	pdfsubject={\imprimirpreambulo},
	    pdfcreator={LaTeX with abnTeX2},
		pdfkeywords={abnt}{latex}{abntex}{abntex2}{trabalho acadêmico}, 
		colorlinks=true,       		% false: boxed links; true: colored links
    	linkcolor=blue,          	% color of internal links
    	citecolor=blue,        		% color of links to bibliography
    	filecolor=magenta,      	% color of file links
		urlcolor=blue,
		bookmarksdepth=4
}
\makeatother
% --- 

% --- 
% Espaçamentos entre linhas e parágrafos 
% --- 

% O tamanho do parágrafo é dado por:
\setlength{\parindent}{1.3cm}

% Controle do espaçamento entre um parágrafo e outro:
\setlength{\parskip}{0.2cm}  % tente também \onelineskip

% ---
% compila o indice
% ---
\makeindex
% ---

% ----
% Início do documento
% ----
\begin{document}

% Retira espaço extra obsoleto entre as frases.
\frenchspacing 

% ----------------------------------------------------------
% ELEMENTOS PRÉ-TEXTUAIS
% ----------------------------------------------------------
% \pretextual

% ---
% Capa
% ---
\imprimircapa
% ---

% ---
% Folha de rosto
% (o * indica que haverá a ficha bibliográfica)
% ---
\imprimirfolhaderosto
% ---

% ---
% Inserir folha de aprovação
% ---
\imprimirfolhadeaprovacao
% ---
% Dedicatória
% ---
\begin{dedicatoria}
   \vspace*{\fill}
   \centering
   \noindent
   \textit{Fazer dedicatória} \vspace*{\fill}
\end{dedicatoria}
% ---

% ---
% Agradecimentos
% ---
\begin{agradecimentos}
Fazer agradecimentos!!

\end{agradecimentos}
% ---

% ---
% Epígrafe
% ---
\begin{epigrafe}
    \vspace*{\fill}
	\begin{flushright}
		\textit{Procurar frase massa!!}
	\end{flushright}
\end{epigrafe}
% ---

% ---
% RESUMOS
% ---

% resumo em português
\setlength{\absparsep}{18pt} % ajusta o espaçamento dos parágrafos do resumo
\begin{resumo}
    \textcolor{red}{A mineração de dados tem se tornado uma área crescente durante os últimos anos, devido ao grande crescimento da quantidade de dados que é gerado e a necessidade do mercado por técnicas que sejam eficazes de reconhecer e extrair o conhecimento destes grandes repositórios de dados de forma automatizada. Como a informação vem desempenhando um papel importante no sucesso e desenvolvimento das grandes organizações como instituições, empresas e entre outros, os métodos de suporte de decisão vem constantemente se tornando presente na realidade dessas organizações, tornando mais seguro as tarefas de colher, tratar, analisar e utilizar as informações resultando em um processo eficaz. Entretanto, apesar dessas informações resumidas e relevantes para a tomada de decisão possuírem um volume menor de dados, elas não se encontram disponíveis no banco, exigindo assim que sejam extraídas a partir de grandes volumes de dados. Com isso, para o processo de extração de conhecimento a partir de grandes bases de dados o KDD surgiu, possuindo em uma de suas etapas a mineração de dados, que utiliza métodos, técnicas e aprendizagem de máquina, aplicando nessa base de dados para poder encontrar padrões em dados que possam auxiliar à tomada de decisão. Nesse contexto, este trabalho apresenta uma aplicação de técnicas de mineração de dados sobre os candidatos ao processo seletivo do vestibular ocorrido em novembro de 2017 da UFRR, com o objetivo de identificar os fatores que influenciam o desempenho da prova dos vestibulandos, e com os resultados obtidos, poderão ser utilizados para traçar os perfis dos candidatos ao processo seletivo do vestibular da UFRR, com a finalidade de levantar informações relevantes que forneçam assistência para a universidade na tomada de decisão.}

 \vspace{\onelineskip}
 
 \noindent 
 \textbf{Palavras-chaves}: ??
\end{resumo}

\begin{resumo}[Abstract]
 \begin{otherlanguage*}{english}
   Fazer Abstract!!

   \vspace{\onelineskip}
 
   \noindent 
   \textbf{Key-words}: ??
 \end{otherlanguage*}
\end{resumo}
% ---
% inserir lista de ilustrações
% ---
\pdfbookmark[0]{\listfigurename}{lof}
\listoffigures*
\cleardoublepage
% ---

% ---
% inserir lista de tabelas
% ---
\pdfbookmark[0]{\listtablename}{lot}
\listoftables*
\cleardoublepage
% ---

% ---
% inserir lista de abreviaturas e siglas
% ---
\begin{siglas}
  \item[PSL] Property Specification Language
  \item[HDL] Hardware Description Language
  \item[VHDL] VHISIC Hardware Description Language
  \item[BMC] Bounded Model Checking
  \item[ESBMC] Efficient SMT-Based Context-Bounded Model Checker
\end{siglas}
% ---

% ---
% inserir lista de símbolos
% ---
% \begin{simbolos}  
%   \item[$ \Gamma $] \todo{Atualizar esta lista!} Letra grega Gama
%   \item[$ \Lambda $] Lambda
%   \item[$ \zeta $] Letra grega minúscula zeta
%   \item[$ \in $] Pertence
% \end{simbolos}
% % ---

% ---
% inserir o sumario
% ---
\pdfbookmark[0]{\contentsname}{toc}
\tableofcontents*
\cleardoublepage
% ---



% ----------------------------------------------------------
% ELEMENTOS TEXTUAIS
% ----------------------------------------------------------
\textual
% ----------------------------------------------------------
% Introdução
% ----------------------------------------------------------
\chapter{INTRODUÇÃO}
%===========================================================
%INTRODUÇÃO
%===========================================================

\par
O sucessivo avanço da tecnologia em nossa sociedade, o número de dados gerados nesses últimos anos cresceu de uma forma explosiva. Segundo \citeonline{Observatory2013}, de 2012 à 2014 foram gerados cerca de 90\% dos dados, sendo que tem como fontes documentos de textos, transações comerciais, streaming de áudios e vídeos, e-mails, telecomunicações, dados pessoais e entre diversas outras. Juntando este crescimento excessivo na quantidade imensa de informações geradas com a grande evolução que é realizado na área de armazenamento de dados, o resultado obtido é uma verdadeira mina de ouro escondida em terabytes ou até mesmo em petabytes de dados \cite{Carvalho2014}. 

\par
Diante desse cenário com uma grande quantidade de informações, surge a necessidade de se utilizar técnicas e ferramentas computacionais para administrar, gerenciar e analisar tais informações. No mundo em que vivemos, atualmente vem crescendo a participação dos computadores na sociedade em inúmeros ramos de atividades como social, científica, saúde e econômica, sendo que já existe computadores que armazenam o que foi efetuado, medido, calculado e decidido desses serviços. Contudo, muito dessas decisões que são tomadas, não se tem o conhecimento suficiente das informações que provém dos dados acumulados em bases de dados de sistemas transacionais \cite{Rabelo2007}.

\par
Com base nisso, para atender esse contexto, surgiu uma nova área conhecida como \textit{Knowledge Discovery in Database} (KDD), uma área da computação que tem como intuito de buscar e extrair informações úteis a respeito dos dados. Em vários casos, esse processo pode ser longo e trabalhoso \cite{Stulp2014}. O KDD é descrito como um processo bastante complexo onde tem como objetivo de extrair o conhecimento gerado de grande quantidade de dados, além de que, na sua estrutura é composto por três etapas principais: pré-processamento, mineração de dados e pós-processamento \cite{Rabelo2007}.

\par
Para muitos, uma das etapas mais importante e complexa de todo o método do procedimento de descoberta do conhecimento é a etapa de mineração de dados (MD). O MD utiliza de ferramentas e técnicas no objetivo de procurar encontrar dados importantes que estão ocultos em uma enorme quantidade de dados absorvidos por um banco de dados para poder dar suporte à tomada de decisão. O processo de mineração de dados segue as seguintes etapas: o reconhecimento do problema, o pré-processamento dos dados, a extração de padrões dos dados obtidos, a definição de uma tarefa e a escolha do algoritmo adequado para a mineração e o pós-processamento que inclui a análise dos resultados obtidos \cite{Stulp2014}. 

\par
A mineração de dados possui várias áreas onde ela é aplicada, uma delas é a mineração de dados educacionais (MDE), que é um campo em desenvolvimento que consiste em examinar uma grande quantidade de dados educacionais com o intuito de obter padrões de dados não triviais. Esse campo utiliza técnicas que verificam padrões em contexto educacional para poder criar parâmetros cognitivos de grande complexidade que proporcione de entender o processo de aprendizado, como por exemplo, na utilização de sistemas que recomendação para aumentar o aprendizado de um determinado estudante durante o seu processo de estudo, no caso, informando em quais conteúdos eles precisam melhorar.

\par
Dessa forma, dentro deste contexto, esta pesquisa, visa criar um modelo de classificação para os candidatos que prestaram o vestibular da Universidade Federal de Roraima,  que tem como intuito de gerar regras para poder criar um perfil e identificar os fatores que causam o mal desempenho do candidato no resultado da sua prova através das técnicas de mineração de dados, aprendizagem de máquina e associação de dados, no que vai contribuir para que a UFRR, MEC e o ministério do desenvolvimento social tome alguma atitude através dos perfis gerados, tanto como avaliar a estratégia do plano de ensino nas escolas como tentar de alguma forma, solucionar os problemas socioeconômicos dos candidatos que foram encontrado.







% %===========================================================
% %MOTIVAÇÃO
% %===========================================================
\section{Motivação}

Recentemente, as universidades têm cada vez mais armazenado dados sobre os alunos e candidatos que participaram do vestibular, isso se deve, ao  grande aumento da redução dos custos e facilidade para manter essas informações. Este aumento da quantidade de dados, no entanto, dificulta uma pesquisa mais precisa, fazendo com que ocorra o risco de o mesmo se transformar em apenas um acúmulo de informações sem utilidade. 

\par
Nesse contexto, o uso do KDD vem ganhando interesse e importância por parte dos donos dessa grande quantidade de dados, pois, as pesquisas obtidas nessa área tem em vista à descoberta de técnicas e tecnologias mais eficientes para a recuperação de dados, procurando encontrar conhecimentos escondido que possam ser úteis, como por exemplo, para as universidades que podem conhecer melhor os seus alunos através dos dados obtido.

No que se torna possível compreender esses dados que são armazenados, no objetivo de solucionar o problema proposto e de se obter um resultado relevante, este trabalho visa em analisar algumas técnicas de mineração proposta na literatura e selecionar a mais adequada no intuito de se criar um modelo de classificação para gerar regras que indicará os fatores que influenciam tais alunos em suas notas do vestibular.


%===========================================================
%DEFINIÇÃO DO PROBLEMA
%===========================================================
\section{Definição do Problema}

\par
Com uma quantidade enorme de dados armazenado dos alunos pelas universidades, muitas não sabem de como utiliza-las para o seu aproveito, consequentemente, muitos desses dados acabam ficando acumulados sem nenhuma utilidade. Contudo, muitos desses dados têm uma grande importância para se obter informações preciosas dos candidatos, sendo que se pode identificar problemas para poder tentar resolver alguns deles. Diante disso, como não existe um padrão consolidado para minerar esses dados, este trabalho visa analisar as técnicas de mineração de dados propostas e selecionar a mais adequada para minerar a base de dados da Comissão Permanente de Vestibular, com intuito de compreender o perfil dos candidatos analisando os fatores que influenciam no seu desempenho no vestibular.

\par
O problema deste trabalho pode então, ser expresso na seguinte questão: \textbf{Como analisar o perfil dos vestibulandos, utilizando uma base de dados que contém fatores socioeconômicos e cadastro geral do candidato, no sentido de criar um modelo de classificação de tal forma que se identifique quais fatores que influenciam no seu desempenho da nota  da prova do vestibular? }


%===========================================================
%OBJETIVOS GERAIS E ESPECIFICOS
%===========================================================

\section{Objetivos}

\subsection{Objetivo Geral}

O objetivo geral deste trabalho é de criar um modelo de classificação através do banco de dados dos candidatos inscritos para o vestibular da Universidade Federal de Roraima, com o intuito de criar regras que se possa gerar um perfil e identificar os fatores que influenciam o desempenho dos candidatos no resultado da nota do vestibular através de técnicas de mineração de dados, aprendizagem de máquina e associação de dados.


\subsection{Objetivos Específicos}

Para alcançar o objetivo geral, os seguintes objetivos específicos necessitam ser executados:


\begin{enumerate}
  \item Seleção e análise da base de dados dos candidatos.
  \item Adotar um algoritmo de aprendizagem de máquina para identificar os fatores que influenciam o desempenho dos candidatos.
  \item Definir e aplicar uma técnica de associação de dados, para consolidar os dados dos fatores de desempenho dos candidatos.
  \item Definir métodos e técnicas para validar os resultados obtidos.
\end{enumerate}

%===========================================================
%METODOLOGIA PROPOSTA
%===========================================================
%\section{Metodologia Proposta}

%Este trabalho utilizou o processo de mineração de dados com técnicas de aprendizagem de máquina para ser aplicado a uma base de dados que contém os dados de cadastro geral e socioeconômico dos candidatos que prestaram o vestibular da Universidade Federal de Roraima no ano de 2017, com a finalidade de extrair conhecimentos relevantes dos candidatos dessa base de dados.

%Para desenvolver este trabalho foram definidas algumas etapas, sendo que a primeira está relacionada mais a fundamentação do trabalho, onde foram conceituados e pesquisados os temas abordados nele, na segunda etapa o foco esteve nas ferramentas e técnicas de mineração e aprendizagem de máquina que podem ser utilizadas, já na terceira etapa consiste na modelagem do projeto e finalmente a última etapa tratou da elaboração da documentação do TCC.



%===========================================================
%CONTRIBUIÇÕES PROPOSTAS
%===========================================================
%\section{Contribuições propostas}

%As contribuições propostas para este trabalho são:
%\begin{itemize}
  %\item O desenvolvimeno de um método para verificação de hardware com o intuito de facilitar os passos da verificação e ao mesmo tempo reduzir substancialmente o tempo de verificação de projetos de hardware descritos em VHDL;
  %\item Este trabalho apresenta para o método proposto, o desenvolvimento e implementação de uma ferramenta de verificação de circuitos lógicos em VHDL com a integração da ferramenta ESBMC (\textit{Efficient SMT-Based Context-Bounded Model Checker})\cite{cordeiro2012smt} na análise.
%\end{itemize}


%===========================================================
%ORGANIZAÇÃO DO TRABALHO
%===========================================================
\section{Organização do Trabalho}
A introdução deste trabalho apresentou: o contexto, definição do problema, objetivos, metodologia e contribuições dessa pesquisa. Os capítulos restantes são organizados da seguinte forma:

\par
No \autoref{chapter:conceitos} \textbf{Conceitos e Definições}, são apresentados os conceitos abordados neste trabalho, especificamente: \textit{Knowledge Discovery in Database} (KDD), Mineração de Dados, Aprendizagem de Máquina e Mineração de Dados Educacionais.

\par
No \autoref{chapter:correlatos} \textbf{Trabalhos Correlatos}, será apresentado os resultados alcançados de outras pesquisas similares, analisando artigos, dissertações e trabalho econclusão de curso e ressaltando a contribuição dos mesmos para o desenvolvimento deste trabalho.

\par
No \autoref{chapter:metodo} \textbf{Método Proposto}, é descrito as etapas de execução do método proposto. Em especial, são descritos a seleção e análise do banco de dados dos candidatos, a aplicação de técnicas de mineração e associação de dados nesse banco e a utilização de métodos para a validação dos resultados obtido.

\par
No \autoref{chapter:cronograma} \textbf{Cronograma}, é apresentado o cronograma de desenvolvimento das atividades que serão realizadas no TCC I e II. %descreve-se a execução de uma avaliação experimental sobre o método proposto, bem como, 
%\textit{benchmarks} utilizados para testes da ferramenta e os resultados obtidos através destes testes.
\par
E por fim no \autoref{chapter:consideracoes} \textbf{Considerações finais}, apresenta-se as considerações finais do trabalho. 


% ----------------------------------------------------------
% Conceitos e Definições
% ----------------------------------------------------------
\chapter{CONCEITOS E DEFINIÇÕES}
\label{chapter:conceitos}
\par
Este capítulo tem como objetivo apresentar os principais conceitos e definições abordados neste trabalho, tais como: Knowledge Discovery in Database (KDD), Mineração de dados, Aprendizagem de Máquina e Mineração de Dados Educacionais.

%============================
% KDD
%============================
\section{Knowledge Discovery in Database (KDD)}

A palvra KDD, sigla de \textit{Knowledge Discovery in Database} que em português é descoberta de conhecimento em bases de dados, foi formalizado em 1989, com intuito de atender os processos referente a busca e extração do conhecimento a partir de bases de dados \cite{Isamir2010}. Segundo \citeonline{Fayyad1996}, KDD é um método formado de várias etapas, não trivial, interativo e iterativo, para o reconhecimento de padrões passível de compreensão, relevantes, novos e potencialmente úteis a partir de grandes grupos de dados.

\par
Considerando a importância de cada componente: \textbf{metódo} se refere a um conjunto de técnicas sequenciais para se conseguir o conhecimento a partir dos dados. \textbf{Padrões} indica unidades que se repetem de maneira previsível. \textbf{Relevante} que tem grau de importância para os padrões descobertos. \textbf{Novos} refere-se as informações não conhecidas sobre os dados, obtidas através dos padrões encontrados. \textbf{Compreensão} indica o nível de entendimento do contexto da solução, que esse padrão pode oferecer aos que utilizarão o conhecimento \cite{Marques2014}.

\par
Segundo \cite{Isamir2010} o processo de KDD extraí e deriva o conhecimento útil dos padrões de dados, que se relacionam por tarefas funcionais, podendo assim, ajudar nas tomadas de decisão. É importante saber que o interativo indica a atuação humana sobre o KDD, no sentido de ele ser o responsável de analisar e interpretar dados e padrões, e o iterativo indica a necessidade de inúmeras vezes a repetição do processo de forma integral e parcial para se obter um resultado satisfatório.

\par
O processo de KDD é constituído por etapas ou fases operacionais organizadas em sequência, como pode ser mostrado na Figura 1:

\begin{itemize}
    \item \textbf{Seleção:} Essa etapa está relacionada a estudar os dados que são disponíveis na base e sua importância, na tentativa de buscar soluções para os problemas identificados \cite{Kampff2013}.
    \item \textbf{Pré-Processamento:} Após da etapa inicial, vem a parte de pré-processamento que tem como objetivo de garantir a qualidade dos dados que estão incluídos no KDD, realizando assim operações básicas como a remoção de ruídos \cite{Dantas2008}.
    \item \textbf{Transformação:} Esta etapa consiste na seleção e transformação dos dados no formato apropriado aos algoritmos e ferramentas de mineração utilizados, como por exemplo em arquivos CSV (\textit{Comma-Separated Values}) \cite{Kampff2013}.
    \item \textbf{Mineração de Dados:} Depois da realização das etapas anteriores, a mineração de dados é iniciada. Ela se constitui na principal etapa do KDD, pois, através dela que as informações ocultas e potencialmente importantes serão extraídas \cite{Kampff2013}. 
    \item \textbf{Pós-Processamento:} Após a mineração de dados começa a avaliação dos resultados, que praticamente envolve a identificação do conhecimento que foi extraído na busca de padrões interessantes, que será utilizado para ajudar na solução dos problemas que incentivaram a mineração \cite{Dantas2008}.
\end{itemize}

\begin{figure}[!htp]
	\begin{center}
    \caption{\label{fig:waveform_fig}Etapas para descoberta de conhecimento adaptada de \cite{Fayyad1996}.}
	\includegraphics[scale=0.55]{Figuras/Etapas_KDD.png}
	\end{center}
    \legend{Fonte:\cite{Kampff2013}}
\end{figure}

\par
Por tanto ao termino da avaliação, o conhecimento extraído na mineração será introduzido e incorporado por outros sistemas, documentando e disponibilizando os métodos, afim  de apresentar o conhecimento obtido ao usuário ou, mais importante de servir como suporte para a tomada de decisão \cite{Kampff2013}. Resumidamente, essas cinco etapas do processo de KDD podem ser classificadas em três categorias: pré-processamento, mineração de dados e pós-processamento, como é apresentado na Figura 2 \cite{Fayyad1996}.

\begin{figure}[!htp]
	\begin{center}
    \caption{\label{fig:waveform_fig}As três partes da divisão do KDD segundo \cite{Fayyad1996}.}
	\includegraphics[scale=0.55]{Figuras/Tres_partes_KDD.png}
	\end{center}
    \legend{Fonte:\cite{Maciel}}
\end{figure}

\par
Como vimos antes, a aplicação de KDD, especificamente na etapa de mineração de dados, podemos perceber a sua enorme importância para todas as áreas que necessitem retirar informações de uma base forte e concreta de dados. Com isso, para a próxima seção, serão apresentados os conceitos relacionado a mineração de dados onde será aprofundado mais sobre essa fase explicando sua metodologia e tarefas de mineração.



%==========================
% MINERAÇÃO DE DADOS
%==========================

\section{Mineração de Dados}

O termo Mineração de dados (MD, do inglês, \textit{Data Mining}, DM), conhecido tambem como uma etapa do KDD, pertence a disciplina que tem como objetivo de encontrar novas informações através da análise de grandes volumes de dados. No caso a frase \textbf{novas informações} está se referindo ao processo de reconhecer novos conhecimentos, assim gerando mais descobertas científicas \cite{Baker2011}.

\par
Segundo \citeonline{Amaral2016}, a mineração de dados seria vários processos para explorar e analisar grandes quantidades de dados na busca de padrões, previsões, erros, associações e entre outros. Em outras palavras, as ferramentas de MD examinam os dados, descobrem oportunidades escondidas ou problemas nos relacionamentos dos dados, e então identificam o comportamento dos negócios, envolvendo a mínima interferência do usuário, fazendo com que ele só se dedique a buscar o conhecimento \cite{Jefferson}. 

\par
Comumente a mineração de dados está relacionada a aprendizagem de máquina, que na área de inteligência artificial desenvolve algoritmos capazes de fazer com o que o computador aprenda a partir do passado, isto é, ele aprende utilizando dados de eventos que já aconteceram \cite{Amaral2016}. Pode-se notar, que as ferramentas de mineração são baseados em algoritmos que formam construção de blocos de inteligência artificial, redes neurais e entre outros, que ajuda a facilitar o trabalho de empresas os auxiliando a maximizar os seus lucros \cite{Jefferson}.

\par
Antes ninguém imaginaria que as aplicações de MD seria tão difundida em diversas áreas, pois, muitas delas não possuíam um modelo de dados armazenados digitalmente \cite{Amaral2016}. Hoje existe inúmeras técnicas e tarefas de mineração de dados que vem sendo utilizados com sucesso, em áreas como \textit{marketing} por exemplo, para saber quais produtos um determinado cliente pretende comprar ou em medicina para prever qual paciente vai contrair um certa doença através do seu histórico \cite{Martinhago2005}.

\par
Além de \textit{marketing} e medicina, segundo \citeonline{Amaral2016} a mineração de dados e a aprendizagem de máquina são aplicadas em outras áreas como processamento de linguagem natural, bioinformática, reconhecimento de fala, detecção de fraude,  finanças, sistemas de recomendação, robóticas, mineração de textos, educação e entre outros. No caso, neste trabalho, a mineração de dados educacional será abordado o seu conceito e importância na seção mais a adiante.

\subsection{Tarefas de Mineração de Dados}

\par

As tarefas correspondem aos problemas que podem ser tratados pela mineração de dados. Dependendo do objetivo que se pretende alcançar a seleção da tarefa deve ser feita para se aplicar sobre a base de dados, existem inúmeras tarefas, mas as que são mais utilizadas pela mineração podem ser classificadas como descrição (não-supervisionado) ou predição (supervisionado) \cite{Garcia2013, Camilo2009}.

\par
Segundo \citeonline{Santos2015}, na predição a análise é feita na forma de encontrar padrões repetidos e generalizados, com a finalidade de achar informações que estão escondidos nos dados. Já na descrição, existe argumentos já formulados onde se tenta procurar respostas que confirmem ou neguem esses argumentos, assim, verificando a sua veracidade. As tarefas mais comuns delas, são: 


\subsubsection{Associação}

\par
Uma tarefa de descrição, consiste em determinar quais itens vão ser adquiridos diretamente em uma mesma transação, isto é, quanto um conjunto de atributos favorece para a presença de outro conjunto \cite{Garcia2013}. No caso se X existir alguma transação, há a possibilidade de Y existir também, pode ser apresentado na forma: SE \textit{atributo} X ENTÃO \textit{atributo} Y \cite{Camilo2009}. 

\par
Segundo \citeonline{Camilo2009}, a associação é uma das tarefas mais conhecidas pelo fato de terem tido ótimos resultados, principalmente na área de marketing com a análise de \textbf{Cesta de Vendas}, onde é identificados quais produtos são levados juntos pelo cliente. Pode-se perceber que são muito utilizado em estudos que tentam descobrir a relação entre os itens, para assim, criarem pacotes de venda ao consumidor \cite{Garcia2013}. 


\subsubsection{Classificação}

\par
Uma tarefa de predição, ela consiste em examinar uma certa característica nos dados (\textit{X}) e atribuir uma classe previamente definida (\textit{Y}), ou seja, visa identificar qual classe um determinado registro pertença, como podemos ver representada na Figura 3 \cite{Garcia2013, Kampff2013}. Exemplos disso é classificar se uma pessoa é de renda baixa, média ou alta, ou classificar se um cliente de um banco é bom ou mau pagador, assim determinando se deve conceder ou não créditos ao mesmo.

\begin{figure}[!htp]
	\begin{center}
    \caption{\label{fig:waveform_fig} Associação entre conjuntos de dados e classes.}
	\includegraphics[scale=0.70]{Figuras/Classificacao.png}
	\end{center}
    \legend{Fonte:\cite{Rabelo2007}}
\end{figure}


\subsubsection{Regressão (Previsão ou Estimação)}

\par
Outra tarefa de predição, muito semelhante a anterior, só que com a diferença de que o atributo especial é identificado com um valor numérico e não categórico. Resume-se na estimativa do valor futuro de algum índice se baseando em dados do comportamento passado desse índice \cite{Camilo2009, LeandroSilva2014}. Exemplo, determinar se o índice da BOVESPA subirá ou descerá amanhã, prever o valor de vida de um equipamento, prever o desempenho do aluno, estimar a quantia a ser gasta por uma família de quatro pessoas durante a volta às aulas e entre outros.

\par
A técnica de regressão pode ser classificada em linear e não-linear. Segundo \citeonline{Camilo2009}, são chamados de linear quando tanto as variáveis preditivas quanto as variáveis de resposta possui um comportamento linear em sua relação, por exemplo, quando um valor Y é uma função linear de X representado na Figura 4. Já a não-linear é o oposto da linear, é quando a relação entre as variáveis de resposta e predição não segue um comportamento linear, no caso, as relações entre as variáveis podem ser representadas na forma de uma função polinomial. 

\begin{figure}[!htp]
	\begin{center}
    \caption{\label{fig:waveform_fig} Gráfico onde o valor y (pontuação alcançada) é uma função linear de x (horas de estudos).}
	\includegraphics[scale=0.99]{Figuras/Regressao_linear.png}
	\end{center}
    \legend{Fonte:\cite{JoaoPaulo}}
\end{figure}

\subsubsection{Agrupamento (Clustering)}

\par
Outra tarefa de descrição, é um método de divisão de um conjunto de dados heterogêneos para um grupo homogêneo \cite{LeandroSilva2014}. Ele tende a reconhecer e aproximar os registros similares, isto é, ele junta um conjunto de registro semelhantes entre si que são diferentes de outros registros nos demais agrupamentos \cite{Camilo2009}. Clustering difere da classificação, pois, não necessita que os dados sejam previamente classificados (aprendizado não-supervisionado).

\par
Segundo \cite{Camilo2009}, o agrupamento não tem a intenção de predizer, estimar ou classificar uma variável, ele somente reconhece os grupos de dados iguais, como é demonstrado na Figura 5. Alguns exemplos de \textit{clustering} como agrupar clientes por região do país ou com comportamento de compra similar, agrupar alunos com desempenho semelhantes, agrupar seções de usuários Web e entre outros.


\begin{figure}[!htp]
	\begin{center}
    \caption{\label{fig:waveform_fig} Registros agrupados em três \textit{clusters}.}
	\includegraphics[scale=0.50]{Figuras/Agrupamento.png}
	\end{center}
    \legend{Fonte:\cite{Camilo2009}}
\end{figure}



\subsection{Mineração de Dados e Aprendizagem de Máquina}

\par
Como foi mencionado em tópicos anteriores, podemos perceber que a área de mineração de dados foi favorecida com diversos conceitos provenientes da área de aprendizagem de máquina, muito desses conceitos são vistos em livros voltado para essa área, tais como as técnicas de classificação, regressão, associação e agrupamento. Os dois grupos principais de mineração de dados, predição e descrição, são parecidos com a forma em que os tipos de aprendizagens são divididos: Aprendizagem supervisionada e não-supervisionada.

\par
Por tanto na próxima seção trataremos de alguns conceitos que já foram citados nesse capítulo, sendo que tais conceitos serão estudados de forma mais precisa, voltada para a área de aprendizagem de máquina.

%==========================
% APRENDIZAGEM DE MAQUINA - tecnicas(arvore de decisão, apriori...)
%==========================
\section{Aprendizagem de Máquina}

\par
Existem diversas atividades relacionadas ao conceito de aprendizagem, dificultando ainda mais a exata definição da palavra, tornando assim dependente de contexto. Porém, no contexto computacional, existe uma definição muito precisa de aprendizagem de máquina que pode ser dada \cite{Henke2011}. Segundo \citeonline{Alpaydin2009}, \textit{Machine Learning} são programas de computador que servem para melhorar um processo de um desempenho, utilizando dados de exemplo ou experiência passada.

\par
Em tese, se tem um modelo definido com alguns parâmetros, onde a aprendizagem é aplicada por um programa de computador para poder otimizar o parâmetro desse modelo utilizando dados de treinamento, sendo que, esse modelo pode ser preditivo, descritivo ou os dois \cite{Alpaydin2009}. No caso, a aprendizagem de máquina utiliza o princípio da estatística na elaboração de seus modelos, porque, a tarefa principal está fazendo a dedução a partir de uma amostra. 

\par
Segundo \citeonline{Henke2011}, os algoritmos de aprendizagem podem ser a solução para resolver diversos problemas, pois, eles podem aprender a determinar padrões das classes que estão envolvidas no problema, através de exemplos existentes obtidos do ambiente. Conforme o modelo genérico da Figura 6 sobre aprendizagem de máquina ocorre o seguinte, o ambiente concede a informação para um elemento de aprendizagem, que utiliza essa informação para melhorar a base de conhecimento para que o elemento de desempenho a use na execução de sua tarefa. 

\begin{figure}[!htp]
	\begin{center}
    \caption{\label{fig:waveform_fig} Modelo genérico de aprendizagem de máquina.}
	\includegraphics[scale=0.45]{Figuras/Modelo_Machine_Learning.png}
	\end{center}
    \legend{Fonte:\cite{Henke2011}}
\end{figure}

\par
Para a compreensão mais abrangente dos algoritmos de aprendizagem de máquina, é importante conhecer o conceito dos termos mais relevantes e mais usados segundo a nomenclatura abaixo de acordo com \cite{Monard2003, Souto2003, Henke2011}.

\begin{itemize}
    \item \textbf{Exemplo (padrão, instância):} É a ação ou o ato de se pegar um fato que já existe para explicar ou usar para alguma situação (por exemplo, para predição). Na maioria dos casos em aprendizagem de máquina, os exemplos são representados por vetores de características. Um exemplo uma amostra de tecido de um paciente.
    \item \textbf{Característica (atributo, variável):} É utilizado na descrição de um padrão (exemplo). Um atributo possui o seu tipo definido em um domínio, que indica os valores que ele pode apresentar (assumir). Exemplo, nível de expressão de um gene do tecido.
    \item \textbf{Vetor de características:} É quando um exemplo é descrito por uma lista de características. Um exemplo que se pode dar é um vetor m-dimensional que descreve para cada m genes, a medida do tecido de um certo paciente.
    \item \textbf{Classe:} Uma classe contém objetos parecidos, enquanto em outra classe possui objetos diferentes da dela. No caso de aprendizagem supervisionado, todo exemplo sempre possui pelo menos uma variável especial que descreve o fenômeno de interesse. Exemplo, as classes poderiam ser a presença ou a ausência de câncer no tecido.
    \item \textbf{Conjunto de exemplos (conjunto de dados):} É formado por uma quantidade de exemplos (padrões) com seus determinados valores de atributos, onde, em aprendizagem de máquina, cada exemplo é relacionado a uma classe. Geralmente, ele é dividido em dois subconjuntos separados: o conjunto de treinamento e o conjunto de teste.
    \item \textbf{Acurácia (taxa de erro):} Seria a porcentagem de erros ou acertos efetuados pelo modelo para um determinado grupo de dados. No geral, a acurácia é aplicada em testes em que em nenhum momento não foram utilizados durante o processo de aprendizagem. Existem outros meios com técnicas mais complexas na estimação da acurácia como \textit{cross-validation} e \textit{bootstrap}. 
    \item \textbf{Falso positivo:} Suponhamos que temos duas classes, A seria positivo e B negativo, o falso positivo seria a quantidade de exemplos da classe B classificados como da classe A. Já o falso negativo seria o oposto, a quantidade de exemplos A classificados como da classe B. Um exemplo bom onde ocorre isso é em matriz de confusão. 
    \item \textbf{Ruído:} Nenhum conjunto de dados é perfeito, é comum em que qualquer ambiente, acabar trabalhando com dados imperfeitos, no caso esses dados imperfeitos são conhecidos como ruído. Então, ele é definido como um conjunto de dados que aparentemente é inconsistente comparado com o restante dos dados existentes. 
    \item \textbf{\textit{Overfitting} (super-ajustamento):} É um termo que é utilizado para descrever quando um modelo se ajusta (especializa) muito bem ao conjunto de dados utilizados no seu treinamento, mas se torna ineficaz para prever novos resultados demonstrando uma taxa de acurácia baixa.
\end{itemize}

\subsection{Tipos de Aprendizagem}

\par
Segundo \citeonline{Henke2011} a aprendizagem de máquina pode ser dividido em dois tipos principais de aprendizagem: aprendizagem com professor e aprendizagem sem o professor.

\subsubsection{Aprendizagem com Professor}

\par
Conhecido também como aprendizado supervisionado, nele se tem a figura de um professor externo, no qual apresenta o conhecimento do ambiente através de um conjunto de exemplos de pares de entrada e saída, onde são propagados em uma sequência de regras que a máquina acompanhara para se obter o efeito desejado \cite{Lorena2007, Henke2011}.

\par
Exemplos desse tipo de aprendizagem, no caso de regressão, dado uma imagem de uma pessoa, através dos dados da imagem fornecido temos que prever a sua idade. Em classificação, dado um exemplo de tumor cancerígeno, temos que prever através da idade e do tamanho do tumor do paciente se ele é maligno ou benigno \cite{Pedro}.

\subsubsection{Aprendizagem sem Professor}

\par
Nesse não existe a presença de um professor, ou seja, não há exemplos rotulados de alguma função para ser aprendida \cite{Lorena2007, Henke2011}. Nessa aprendizagem são identificados dois tipos de subdivisão:

\begin{itemize}
    \item \textbf{Aprendizagem por reforço:} Não se sabe qual a saída correta. Ele se preocupa de como um agente deve agir em um ambiente de forma que potencialize alguma noção de recompensa a longo tempo. Como os pares de entrada e saída não são fornecidos, ele tenta encontrar uma maneira de mapeá-los através de uma interação contínua com o ambiente, para poder reduzir um índice escalar de desempenho \cite{Henke2011, Alpaydin2009}. 
    \item \textbf{Aprendizagem não-supervisionada:} Não existe padrões de saída desejada. O algoritmo aprende a agrupar ou representar as entradas submetidas de acordo com uma medida de qualidade. Esse tipo de aprendizagem é usada principalmente quando se quer encontrar padrões ou tendências que ajudem na compreensão dos dados, nela inclui estimação de densidade, formação de agrupamentos e dentre outros \cite{Henke2011, Lorena2007}.
\end{itemize}

\par
Exemplo de aprendizagem não-supervisionada, no caso de agrupamento, dado um determinado conjunto de 1000 pesquisas de uma universidade, queremos encontrar um jeito de agrupar automaticamente elas de acordo com suas semelhanças ou relações por diferentes variáveis, tais como frases, sequências das palavras, contagens de páginas e etc \cite{Pedro}. Já na aprendizagem por reforço, temos como exemplo jogos, robôs e navegação.


\subsection{Algoritmos de Aprendizagem}

\par
Existe uma extensa quantidade de algoritmos disponíveis para classificação/regressão em aprendizagem de máquina, sendo que serão falados apenas alguns dos principais que são mais utilizados. Todos os algoritmos que serão mencionados abaixo, podem ser utilizados tanto para classificação quanto para regressão. %\textcolor{red}{Avore de decisão - supervisionado, KNN - supervisionado, redes neurais - supervisionado, naive bayes - supervisionado, SVM - supervisionado.}


\subsubsection{Árvore de Decisão}

\par
Árvore de decisão (\textit{Decision Trees}) é uma ferramenta de aprendizagem que pode ser utilizada para tomada de decisão e dedução de valores categóricos. A ideia é de um aprendizado indutivo: se cria uma hipótese constituída em instâncias específicas que gera conclusões gerais \cite{Marques2014}. As árvores de decisão pegam como entrada um caso representado por um conjunto de atributos que retorna a uma decisão, que é a categoria para o valor de entrada


\par
Uma árvore é composta por três elementos importantes: os nós internos representam diferentes características, os ramos entre esses nós representam a possíveis valores que essas características podem possuir, enquanto os nós folhas representam o resultados, no caso, os valores de classificação da aplicação \cite{Henke2011, Barros2012}. Segundo \citeonline{Barros2012}, para a realização da classificação, deve ter uma tupla correspondente às características dos fluxos, que percorre pela árvore do nó raiz até  todos os nós folhas.  

\par
Para o melhor entendimento do processo de classificação de uma árvore de decisão, a Figura 7 mostra um exemplo usado no treinamento de dados para a tarefa de detecção de intrusão. De início, uma base de dados é fornecida ao algoritmo, onde, ele gera uma árvore feita para separar as amostras da classe Maligna de amostras da classe Benigna. Depois da construção da árvore, os dados novos podem ser fornecidos ao classificador, ao qual concedera uma das duas classes à amostra testada \cite{Henke2011}. 

\begin{figure}[!htp]
	\begin{center}
    \caption{\label{fig:waveform_fig} Treinamento dos dados em uma Árvore de Decisão.}
	\includegraphics[scale=0.50]{Figuras/Arvore_decisao.png}
	\end{center}
    \legend{Fonte:\cite{Henke2011}}
\end{figure}

\par
Pelo que podemos perceber em um ponto de vista de decisão de negócios, uma árvore de decisão é uma quantidade mínima de perguntas sim ou não que alguém tem que perguntar, para poder avaliar a possibilidade de fazer uma decisão correta a maior parte do tempo. Como ferramenta, possibilita tratar o problema de forma estruturada e sistemática para chegar a uma conclusão lógica \cite{Sara}. 


\subsubsection{K-\textit{Nerarest Neighbor} (KNN)}

\par
O algoritmo de KNN ou K-Vizinho Mais Próximo é um potente algoritmo não-paramétrico de classificação e regressão, utilizado desde 1950 na área da estatística \cite{Carvalho2014}. Segundo \citeonline{Henke2011}, ela é uma técnica baseada em instâncias, que constitui em atribuir a classe de cada elemento novo a partir da classe dominante conseguida através de seus vizinhos mais próximos, detectado no conjunto de treinamento. A definição de vizinhança é feita segundo a uma medida de distância estimado no espaço de atributos, exemplo de medida é a Euclidiana. 

\par
Por exemplo, queremos determinar a renda de uma pessoa de uma região, pesquisando k=20 vizinhos mais próximos desse indivíduo, podemos obter a renda através de valores dos atributos bairro, moradia, profissão, escolaridade e idade. Um de alguns problemas dessa técnica é a necessidade de se ter um número de atributos bastante razoável para a determinação da vizinhança, e também, esse processo de classificação pode ser bastante cansativo para a máquina quando o conjunto de treinamento tem bastante dados \cite{Cortes2002, Henke2011}.

\subsubsection{Naive Bayes}

\par
Essa técnica de aprendizagem de máquina se baseia em fundamentações simples, mas que constantemente gera uma alta taxa de classificação em questões reais, como por exemplo, uma resposta para tal questionamento, como, qual a chance de um ataque ser de um determinado tipo, dado alguns acontecimentos do sistema observado? \cite{Henke2011}. Segundo \citeonline{Barros2012}, naive bayes analisa as categorias de aplicação para cada instância e relações dos atributos, derivando a probabilidade para cada relação delas entre seus valores.

\par
Segundo \cite{Henke2011}, uma de suas principais vantagens é a baixa complexidade na fase de treinamento, sendo que essa fase envolve somente os cálculos de frequência para que as probabilidades sejam conseguidas. Em aplicações \textit{online}, como problemas envolvendo segurança de redes, cujo, treinamento deve ocorrer online e com frequência continua, por ela ter essa característica naive bayes é muito indicada. Outra característica importante para segurança de redes, é o fato de ela poder manipular atributos nominais e numérico.

\par
O modelo de Naive Bayes é simples de construir e principalmente útil para grandes volumes de dados. Além de fácil, ele é conhecido também por ganhar de técnicas de classificação que são mais sofisticadas \cite{Sunil}. Naive Bayes utiliza o teorema de bayes que fornece uma forma de calcular a probabilidade posterior P(A|B) a partir de  P(A|B), P(A) e P(B). Observe a \autoref{eq:Bayes} abaixo:

\begin{equation}
    \label{eq:Bayes}
        {P(A|B)\quad =\frac { P(B|A)\quad *\quad P(A) }{ P(B) } }
\end{equation}

\par
Segundo \citeonline{Thiago}, precisamos dos seguintes dados utilizado na \autoref{eq:Bayes}, que são:
\begin{itemize}
    \item \textbf{P(A|B):} É a probabilidade posterior da classe (B, alvo) dada preditor (A, atributos), ou seja, a probabilidade de A acontecer dado que B ocorreu.
    \item \textbf{P(B|A):} É a probabilidade que representa a probabilidade de preditor dada a classe, ou seja, a probabilidade de B acontecer dado que A ocorreu.
    \item \textbf{P(A):} É a probabilidade original da classe, ou seja, a probabilidade de A ocorrer.
     \item \textbf{P(B):} É a probabilidade original do preditor, ou seja, a probabilidade de B ocorrer.
\end{itemize}

\subsubsection{Redes Neurais}

\par
Redes Neurais é uma técnica originado da psicologia e neurobiologia, que compõe basicamente de simulações baseada no comportamento dos neurônios \cite{Camilo2009}. Tal ideia vem da seguinte razão: o cérebro é tipo um aparelho que processa a informação com inúmeras habilidades, tais como por exemplo, o reconhecimento de voz, visão e etc. Por tanto, uma rede neural é bastante conectada por várias unidades (nó) como os neurônios, do qual a saída é uma combinação de várias entradas para outros neurônios \cite{Henke2011, Barros2012}.

\par
Segundo \cite{Carvalho2014}, na área de aprendizagem de máquina os algoritmos de redes neurais artificial são considerados poderosos, sendo que, são usados na resolução de problemas lineares e não-lineares, tendo apoio tanto no aprendizado supervisionado quanto no não supervisionado. Sua utilização mais constante é na solução de problemas não-lineares de classificação, através de uma estrutura conhecida com \textit{Multi-Layer Perceptron} (MPL) que sempre trabalha em conjunto nessa área de aprendizagem com o algoritmo mais conhecido, o algoritmo de Retropropagação (\textit{Backpropagation)}.

\par
Em bases de dados que possuam muitos ruídos, que no caso são as inconsistências encontradas no conjunto de dados de treinamento, a técnica de redes neural é bastante recomendada. Pois, eles conseguem ser relativamente quase imunes a este ruídos, diferentes de outros modelos de aprendizagem supervisionado como árvore de decisão que são severamente afetados por esses ruídos \cite{Carvalho2014}. A Figura 8 logo depois, apresenta as diversas camadas que podem ser criadas em um processamento de uma rede neural \cite{Cortes2002}.

\begin{figure}[!htp]
	\begin{center}
    \caption{\label{fig:waveform_fig} Representação de um processamento de uma rede neural.}
	\includegraphics[scale=0.56]{Figuras/Redes_neural.png}
	\end{center}
    \legend{Fonte:\cite{Cortes2002}}
\end{figure}

\par
Segundo \citeonline{Cortes2002}, apresentado na Figura 8, todas as camadas intermediárias simbolizam os diferentes níveis de conhecimento que são obtidos no processo de seu percurso, numa tentativa de copiar o cérebro humano. Apesar de ser uma técnica poderosa, redes neurais é pouco utilizado em mineração de dados, pelo fato de seus algoritmos serem em sua maioria, incompreensíveis e também a função de aprendizado ser consideravelmente lenta, devido as várias interações que o algoritmo de retropropagação realiza para atualizar os valores da rede \cite{Carvalho2014}. 



\subsubsection{\textit{Support Vector Machines} (SVM)}

\par
Máquina de vetores de suporte é um algoritmo supervisionado que é usado para o trabalho de classificação, onde ele utiliza um hiperplano como separador de classes. Este hiperplano é encontrado através da utilização do conjunto de treinamento (vetores de suporte) e ele trabalha como uma base para o limite da decisão ao classificar \cite{Costa2012}. Segundo \citeonline{Henke2011}, ela é uma técnica de classificação bastante aplicada em problemas de segurança, como por exemplo na detecção de \textit{phishing} e detecção de intrusos.

\par
O funcionamento do SVM pode ser dito da seguinte forma: se tem duas classes e um conjunto de treinamento onde as amostras pertencem as classes, a máquina vetor de suporte cria um hiperplano (conhecido como hiperplano de separação ótima) que divide o espaço de atributos em duas regiões, potencializando a margem de separação entre as mesmas. As amostras que antes não eram conhecidas são mapeadas para esse mesmo espaço e depois atribuídas a uma das classes \cite{Henke2011}.

\begin{figure}[!htp]
	\begin{center}
    \caption{\label{fig:waveform_fig} Dados de treinamentos representando alunos que passaram (círculos brancos) e não passaram (círculos cinzas) em uma disciplina.}
	\includegraphics[scale=0.60]{Figuras/SVM.png}
	\end{center}
    \legend{Fonte:\cite{Costa2012}}
\end{figure}

\par
Como exemplo acima na Figura 9, suponhamos que os dados de treinamento sejam referentes aos alunos de uma turma com suas informações, apresentados como círculos, como a quantidade de postagens em um fórum de discussão (variável preditora). Além que, os dados representam cada aluno conforme o seu desempenho na disciplina (variável preditiva), alunos que passaram (círculos brancos) e que reprovaram (círculos cinzas). A priori, a meta do SVM é encontrar a melhor maneira de separar os dois grupos de alunos \cite{Costa2012}.



\par
Pode-se notar que existe uma quantidade infinita de hiperplanos que são representas na Figura 9 como essas linhas tracejadas, que podem separar as classes demonstradas que no caso são os círculos brancos e círculos cinzas. Então o propósito da máquina vetor de suporte é de encontrar o melhor hiperplano que maximize a distância entre elas das classes vizinhas. Um exemplo do melhor hiperplano para os dados que foram mostrado na Figura 9 encontrado pelo SVM é apresentado na Figura 10 \cite{Costa2012}. 


\begin{figure}[!htp]
	\begin{center}
    \caption{\label{fig:waveform_fig} Representação do melhor hiperplano que separa as classe, encontrado pelo algoritmo de SVM.}
	\includegraphics[scale=0.60]{Figuras/SVM_2.png}
	\end{center}
    \legend{Fonte:\cite{Costa2012}}
\end{figure}


\par
Embora SVM seja uma técnica recente, ela tem atraído muita atenção pelos seus resultados como: a possibilidade de modelar casos não-lineares difíceis gerando modelos de simples compreensão, consegue altos índices de assertividade, pode ser usada tanto para relações lineares quanto relações não lineares e entre outros \cite{Camilo2009}. Ultimamente um dos problemas dessa técnica é o tempo que se usa no aprendizado e muitas pesquisas estão se concentrando neste aspecto. 




%==========================
% MINERAÇÃO DE DADOS EDUCACIONAIS
%==========================

\section{Mineração de Dados Educacionais}

\par
Há pouco tempo, com o crescimento dos cursos a distância e também com aqueles que usam suporte computacional, muitos cientistas da área de Informática na Educação, têm demonstrado interesse em usar a mineração de dados para solucionar perguntas científica na área da educação \cite{Baker2011}. Nesse contexto, uma nova área de pesquisa surgiu denominada como Mineração de Dados Educacionais (MDE).

\par
A área de MDE (do inglês, \textit{Educational Data Mining}, EDM), é uma área que tem como proposito de desenvolver ou adaptar técnicas e algoritmos de mineração de dados existentes na literatura, de tal modo que sirvam a entender melhor os dados educacionais, gerados principalmente por professores e estudantes, tais como AVAs (Ambiente Virtual de Aprendizagem), STIs (Sistemas de Tutores Inteligentes), entre outros \cite{Costa2012, Marques2014}.

\par
Segundo \citeonline{Coutinho2016}, a MDE tem como apoio de disponibilizar a descoberta de conhecimentos que sejam importantes, único e útil, tais como: identificar padrões entre alunos, analisar o desempenho através de predição e a identificar perfis de forma que auxilie a gestão qualitativa do ensino a distância (EAD). Os trabalhos nessa área, estão bastante concentrados em situações relacionados a um tipo específico de instituição, no caso, as Instituições de Ensino Superior (IES).

\par



% ----------------------------------------------------------
% Revisão de Literatura
% ----------------------------------------------------------
\chapter{TRABALHOS CORRELATOS}
\label{chapter:correlatos}

\par
Neste capítulo serão apresentados alguns trabalhos relacionados com este, que envolvem mineração de dados aplicado na área educacional, com o objetivo de extrair informações contida no banco de dados dos candidatos que prestaram a prova do ENEM ou a prova do vestibular de alguma universidade.

\section{Mineração de Dados Educacionais nos Resultados do ENEM de 2015}

\par
O trabalho de \citeonline{Simon2017} tem como objetivo em gerar um modelo preditivo da base de desempenho médio na área de ciências da natureza e suas tecnologias dos alunos de escolas do ensino médio, baseado nos dados públicos obtidos do Exame Nacional do Ensino Médio (ENEM) de 2015. Tal motivação foi pelo fato do cenário preocupante do baixo desempenho dos alunos do ensino básico no Brasil, conforme os dados de 2015 do PISA (\textit{Programme for International Students Assessment}).

\par
Segundo \citeonline{Simon2017} foi feito o pré-processamento dos dados do ENEM fornecidos pelo INEP, sendo que para cada área avaliada apresentavam-se 25 colunas que representavam uma informação ou variável, onde cada uma delas eram divididas em três categorias: características da escola, indicadores e desempenho. Das 25 variáveis, apenas 9 foram selecionadas como importante para a mineração de dados conforme o indicador da coluna Id representado na Figura 11. 

\begin{figure}[!htp]
	\begin{center}
    \caption{\label{fig:waveform_fig} Variáveis disponibilizadas pelo ENEM por escola.}
	\includegraphics[scale=0.99]{Figuras/Tabela_ENEM.png}
	\end{center}
    \legend{Fonte:\cite{Simon2017}}
\end{figure}

\par
\citeonline{Simon2017} destacaram que a variável Média Escola indicava o desempenho médio dos alunos da escola na área de ciências da natureza e suas tecnologias, onde esses dados eram fornecidos como valor numérico pelo INEP, então para facilitar a mineração dos dados, ele teve que converte-lo para uma categoria mais limitada que foi dividida em quatro valores possíveis, segundo a escala prevista pelo INEP conforme a Figura 12. Os dados fornecidos pelo INEP, vinham em uma planilha no formato .xlsx e teve que ser convertido para .csv, pois, o formato não era suportado pelo software WEKA.


\begin{figure}[!htp]
	\begin{center}
    \caption{\label{fig:waveform_fig} Categoria para os valores Média Escola.}
	\includegraphics[scale=0.50]{Figuras/Tabela_ENEM_2.png}
	\end{center}
    \legend{Fonte:\cite{Simon2017}}
\end{figure}

\par
Relacionando os fatores socioeconômicos e o índice de desempenho médio em ciências da natureza e suas tecnologias dos alunos, \citeonline{Simon2017} construiu um modelo preditivo utilizando a técnica de mineração de árvore de decisão. Tal técnica, usou o algoritmo j48 através do software WEKA, que foi executado utilizando a opção de \textit{cross-validation}, onde o valor para \textit{fold} era igual a 10 e com a variável dependente Média Escola. Com a árvore obtida, demonstrou como variável independente mais relevante foi a variável Tipo Escola, que se divide entre quatro valores: privada, federal, estadual e municipal.

%Variavel dependete seria o resultado e a variavel independente seria o caminho que ele iria percorre até o resultado. Exemplo: variavel dependete (aluno passou | aluno não passou), variavél independente(entregou os trabalhos, participou das aulas, tirou nota boa na prova, numeros de faltas e etc).

\par
Depois da variável Tipo Escola, a variável hierarquicamente seguinte é a Nível socioeconômico e a partir dela as variáveis seguintes se alternam de posição conforme as funções dos valores dos níveis superiores. Como resultado, sobre o desempenho médio dos alunos na área de ciências da natureza e suas tecnologias, as categorias que mais se destacaram foram a III e IV, acima de 550 pontos, que ocorreram nas escolas: privadas e estaduais com o nível socioeconômico muito alto, federais com o nível médio alto e muito alto e municipal com o nível médio alto.

\par
Em comparação com o trabalho proposto neste TCC,  a base de dados dos fatores socioeconômico e outras informações fornecidos do ENEM pelo INEP, que está relacionado na parte de pré-processamento, é semelhante aos dados fornecidos pela Comissão Permanente do Vestibular (CPV) da UFRR, como base pretende-se adotar as mesmas variáveis que foram escolhidas no trabalho de \citeonline{Simon2017} para agilizar o processo de minerar os dados mais importante para solucionar o problema que é apresentado neste trabalho.

\par
Outro ponto importante é o modelo preditivo que foi abordado, que é a técnica de mineração de Árvore de Decisão que utiliza o algoritmo J48 do software WEKA, já que neste trabalho será utilizado um conceito parecido, contudo será usado um modelo de classificação, outro ponto, seria de utilizar a opção de \textit{cross-validation}, no caso o K-\textit{folds}, que foi usado no trabalho de \citeonline{Simon2017} para os treinos e testes dos dados para a validação do modelo.



\section{Prática de Mineração de Dados no Exame Nacional do Ensino Médio}

\par
O trabalho de \citeonline{Silva2014}, apresenta um estudo de mineração de dados educacionais, mais especificamente, utilizando a tarefa de Associação de Dados de MD no intuito de encontrar padrões de regras nos dados dos questionários socioeconômicos e resultados das provas do Exame Nacional de Ensino Médio (ENEM), fornecidos pelo INEP. O objetivo é de executar as etapas do KDD para o processamento dos dados do ENEM, de modo que por consequência, se utilize a técnica de associação para se encontrar regras proposicionais que relacionam os fatores socioeconômicos do candidato com o seu desempenho na prova.

\par
Para o trabalho de \citeonline{Silva2014}, foi utilizado o banco de dados com os questionários socioeconômico e os desempenhos da prova do ENEM de 2010, fornecido pelo INEP, onde, para acessar os dados do banco dele, foi usado os softwares Oracle Express Edition 11g e PL/SQL Developer que permitem a extração dos dados que foram determinado para o foco da pesquisa. Os dados utilizados para fazer a mineração de dados, foi da região Sudeste das suas capitais: Vitória, Belo Horizonte, São Paulo e Rio de Janeiro.

\par
Na parte de pré-processamento, foi selecionado somente os dados das capitais do Sudeste que resultaram em 452.710 alunos, que entretanto, foram eliminados 310.000 alunos das quatros capitais, pois, eles não compareceram no dia da prova. As questões selecionadas para a pesquisa  foram de quantas pessoas moravam com o candidato, a renda familiar mensal, o nível de escolaridade da mãe e o tipo de escola que cursou no ensino médio. Sendo que essas questões foram analisadas para saber se contribuía, interferia ou afetava a nota e o desempenho do candidato na prova.

\par
Baseado nas respostas das questões analisadas, a nota da prova objetiva foi classificada conforme o conceito mostrada na Figura 13. Para a preparação da regra de associação, \citeonline{Silva2014} fez uma limpeza e integração nos dados, alterando as colunas principais para binominais, quer dizer, em 0 e 1, sendo 1 a opção que foi respondida no questionário e 0 para o restante das opções que não foi escolhida no mesmo questionário, isso aplica também a categoria da prova, como podemos ver representado na Figura 14, onde o aluno não foi inserido sendo três de quatro delas. 


\begin{figure}[!htp]
	\begin{center}
    \caption{\label{fig:waveform_fig} Relação de tranformação da nota em conceito.}
	\includegraphics[scale=0.50]{Figuras/Prova_objetiva.png}
	\end{center}
    \legend{Fonte:\cite{Silva2014}}
\end{figure}

\begin{figure}[!htp]
	\begin{center}
    \caption{\label{fig:waveform_fig} Amostragem de alguns dos dados limpos e integrados.}
	\includegraphics[scale=0.49]{Figuras/Dados_limpos_integrados.png}
	\end{center}
    \legend{Fonte:\cite{Silva2014}}
\end{figure}

\par
Observando a segunda linha da Figura 14, percebemos que o candidato é da cidade de Belo Horizonte, mora no total de sete pessoas contando com ele, a mãe dele possui o 1º grau de nível de escolaridade e que a classificação da nota da prova objetiva dele é regular. Para o melhor entendimento, \citeonline{Silva2014} especifica os critérios das colunas, que são:

\begin{itemize}
    \item \textbf{NU\_INSCRICAO:} O número de identificação do candidato. 
    \item \textbf{UFXX:} A sigla da cidade onde o candidato mora.
    \item \textbf{QXXABC:} Sendo XX a questão escolhida de acordo com o questionário socioeconômico, ABC representa alternativa escolhida da questão de forma abreviada.
    \item \textbf{NOTAXXX:} Sendo XXX a classificação da nota da prova que o candidato conseguiu. 
\end{itemize}

\par
Para os experimentos que foram feitos, \citeonline{Silva2014} utilizou o software RapidMiner 5.1 que usa o código aberto Java, e para a tarefa de associação de dados, é usado o algoritmo apriori que na etapa de verificação de itens conjuntos ele faz um processo interativo para a combinação de itens. Este algoritmo realiza busca por largura e é definido em três características: suporte mínimo, confiança mínima e K-itemset. Lembrando que o suporte é a frequência o qual uma regra é aplicada em um determinado conjunto de dados e a confiança mede a confiabilidade da interferência feita por uma regra. 

\par
Conforme a Figura 15, \citeonline{Silva2014} observou que a medida que o suporte e a confiança diminuía, o número de regras aumentava.   

\begin{figure}[!htp]
	\begin{center}
    \caption{\label{fig:waveform_fig} Simulação de Suporte e Confiança.}
	\includegraphics[scale=0.49]{Figuras/Simulacao_suporte_confianca.png}
	\end{center}
    \legend{Fonte:\cite{Silva2014}}
\end{figure}

\par
Das quatros simulações, em cada uma se analisou o seguinte: na simulação 1, a regra aplicada era se o aluno era de escola publica então a sua nota prova seria regular (76\% de confiança, ocorre 53\% dos alunos) e a outra regra se o aluno tirou nota regular na prova então ele é de escola pública (80\% de confiança, ocorre 53\% dos alunos). Na simulação 2 a primeira regra com confiança mais baixa gerada era se o candidato morasse com quatro e sete pessoas então ele era de escola pública (74\% de confiança, ocorre 31\% dos alunos) e a última regra com maior confiança era se os pais tivessem escolaridade de até o primeiro grau então ele era de escola pública (89\% de confiança, ocorre 36\% dos alunos).

\par
Na simulação 3, a primeira regra gerada de menor confiança é se os pais tivessem escolaridade de até o 1º grau então o candidato é de escola pública e a nota da prova é regular (71\% de confiança, ocorre 28\% dos alunos), e a última regra com maior confiança e se o candidato é de escola pública e possui pais com escolaridade de até o 1º grau, então sua nota da prova é regular (90\% de confiança, ocorre 19\% dos alunos). Na simulação 4, \citeonline{Silva2014} observou que alguns atributos como escolaridade dos pais (2GR) e a renda familiar levava (TRE) o candidato a obter uma nota regular (REG) ou vim de uma escola pública (PUB), como podemos perceber na última regra com maior confiança na Figura 16.

% Podemos evidenciar na penúltima regra que, se o aluno tirou nota regular na prova, os pais tiveram escolaridade até o primeiro grau e a renda da família for de um a três salários mínimos então o aluno estudou em escola pública com 92% de confiança, ocorrendo em 19% dos alunos

\begin{figure}[!htp]
	\begin{center}
    \caption{\label{fig:waveform_fig} Simulação 4, Suporte Mínimo 25\% e Confiança Mínima 70\%.}
	\includegraphics[scale=0.49]{Figuras/Simulacao_quatro.png}
	\end{center}
    \legend{Fonte:\cite{Silva2014}}
\end{figure}

\par
Como representado na Figura 17, \citeonline{Silva2014} verificou que ao diminuir o valor do suporte mínimo, há um acréscimo em casos encontrados na base de dados que utiliza associação de dados, em que se tem regras mais verdadeiras, chegando pelo menos a 90\% de confiança, que são geradas na simulação correspondentes. Para a demonstração, \citeonline{Silva2014} criou faixas de confianças que começava a partir de 70\%, onde, cada faixa era verificada entre as simulações com a quantidade de regras que eram geradas, como por exemplo, na simulação 1, dentre as duas regras, se tinha de 76\% e 81\% de confiança, então ele se enquadra na faixa de 70\% até 80\%, totalizando 50\%.

\begin{figure}[!htp]
	\begin{center}
    \caption{\label{fig:waveform_fig} Sintetização das simulações.}
	\includegraphics[scale=0.49]{Figuras/Sintetizacao_simulacoes.png}
	\end{center}
    \legend{Fonte:\cite{Silva2014}}
\end{figure}

\par
Contudo, nas regras de maior confiança obtidos com base nas duas últimas simulações, \citeonline{Silva2014} verificou que candidatos de São Paulo que moram com quatro a sete pessoas, com renda familiar de até três salários mínimos, com pais de escolaridade de até 1º grau e com a nota regular obtida no ENEM, então ele é de escola pública, o que reforça que a educação das escolas públicas das capitais da região sudeste está em um nível baixo. Então \citeonline{Silva2014} conclui que com o conhecimento obtido a partir dos resultados, como alta quantidade de pessoas que moram com o candidato, baixa renda familiar e pais com escolaridade de nível primário, são atributos que afetam o baixo desempenho do candidato. 

No trabalho de \citeonline{Silva2014}, demonstra casos semelhantes que pode ocorrer no trabalho proposto, como por exemplo, utilizar dados obtidos do questionário socioeconômico pelo INEP (que neste caso será pelo CPV), para poder relacionar qual destas questões é um dos motivos que afetam o desempenho do candidato na prova, outro ponto, é a eliminação de ruído de dados na parte de pré-processamento, já que nem todos os candidatos inscritos para o vestibular, por algum motivo, não comparecerão para fazer a prova, então será necessário tirar os candidatos que não participaram.

\par
O algoritmo que \citeonline{Silva2014} utilizou para a mineração de dados foi o apriori, que faz uma verificação dos dados através de um processo interativo para combina-los, para então se obter um resultado. Como observado, poderia utilizar como exemplo as técnicas, algoritmo e as questões escolhidas do questionário socioeconômico abordados pelo \citeonline{Silva2014} em conjunto com as outras técnicas e outros conceitos apresentado neste trabalho, para poder assim, obter mais exatidão e aprimoramento nos resultados obtidos.

\par
Um possível problema que encontraria, caso fosse se basear nos exemplos de \citeonline{Silva2014}, é que as questões do questionário socioeconômico fornecidas pelo INEP vem com alternativas de escolha para os candidatos responderem, o que é diferente das questões fornecido pelo CPV, apenas algumas delas vem com alternativa as outras questões vem com resposta direta, um exemplo de uma das questões do questionário socioecômomico do ENEM, é a questão sobre a quantidade de pessoas que moram com o candidatos, onde se tem as seguintes alternativas: (a) três pessoas, (b) quatro pessoas, (c) sete pessoas e (d) mais de sete pessoas.

\section{Descoberta de Conhecimento Sobre o Processo Seletivo da UFPR}

\par
O trabalho de \citeonline{Martinhago2005}, tem como objetivo geral de traçar o perfil dos candidatos ao processo de seleção para a admissão no ensino superior da Universidade Federal do Paraná (UFPR) do campus de Curitiba, utilizando a técnica de mineração de dados Árvore de Decisão através dos algoritmos de classificação J48.J48 e J48.PART, implementados pelo software WEKA. \citeonline{Martinhago2005} utilizou a base de dados dos questionários sócio educacional realizado durante a inscrição do vestibular feito em 2003, os dados do cadastro geral e os dados das notas obtidas na prova e redação, em conjunto com a nota, média das notas e os status (resultado do vestibular) do candidato pelo ENEM. 

\par
Na fase de Pré-processamento, na parte de seleção de dados, \citeonline{Martinhago2005} juntou os 31 itens do questionário sócio educacional com os 24 itens do cadastro geral, para formar uma planilha com base de dados de 55 itens (atributos), para servir como base referencial para aplicação das técnicas de mineração. Ao analisar a base de dados, foi detectado inúmeros itens de dados em branco, valores absurdos ou erro de digitação, logo, \citeonline{Martinhago2005} teve que preencher esses dados com a letra N (NULL), pois, a ferramenta WEKA que foi utilizado, necessita que todos os registros estejam preenchidos. Também foi eliminado atributos que eram repetidos por causa da junção das duas planilha e dados que eram irrelevantes para a pesquisa, tais como protocolo, bairro, CEP e etc.

\par
Na fase de Mineração de Dados, \citeonline{Martinhago2005} utilizou a ténica de classificação de mineração, a Árvore de decisão, utilizando o algoritmo J48 através do software WEKA, para a criação do modelo preditivo aplicado na base de dados obtidos de 46.532 candidatos. No momento, foram realizados testes onde foi selecionado alguns atributos, tais como, idade, sexo, notas e etc., nos testes seguinte, foi utilizado os mesmos atributos dos testes anteriores em conjunto com alguns atributos culturais e sócio educacional, tais como, o turno cursado (se fez cursinho ou não), o tipo de escola e etc. Sucessivamente, foram realizados outros testes com outros atributos.

\par
Utilizando o algoritmo J48, em cada execução, várias regras foram geradas e o resultados obtidos para cada base de dados analisados foi o seguinte: para a base de dados dos candidatos ao 11 cursos mais concorridos, \citeonline{Martinhago2005} observou que o conjuntos de notas categorizados com valores acima da faixa de 5 pontos, influenciam na aprovação do candidato, principalmente as notas de redação e língua portuguesa e outras em geral, outra regra relevante, foi de candidatos com notas na faixa de 6 pontos (4,208 a 8,969 pontos) ou acima no ENEM, alcançaram sucesso do vestibular, um exemplo, candidatos para o curso de Direito que obtém nota em Química, Biologia e Matemática acima da faixa de 5 pontos (4,91 a 9,80), segundo \citeonline{Martinhago2005} geralmente tem chance de ser classificado após a análise da redação.

\par
Para o curso mais concorrido, que no caso foi Medicina, segundo \citeonline{Martinhago2005} as notas obtidas nas provas do vestibular e o do ENEM, foi o principal fator que influenciou na aprovação do candidato ao contrário dos fatores sócioeconômico e culturais, que não foram considerados relevantes nos testes realizados, sendo que nas pesquisas do INEP, fatores como renda familiar, escolaridade dos pais, ter feito cursos de pré-vestibular e ter estudado em escola particular ou pública, seriam determinantes para influenciar a aprovação do candidato no vestibular para o curso de Medicina, porém, mesmo não sendo os principais fatores que influenciam, ainda sim as condições sócioeconômicas interferem na aprovação desse curso. 

\par
Para a base de dados dos candidatos aos onze cursos menos concorridos, através das regras geradas nos testes, segundo \citeonline{Martinhago2005} a notas que influenciaram na aprovação dos candidatos foi em Geografia, Redação, Língua Estrangeira e História e as notas que não foram importantes para aprovação eram Matemática, Química e Física. \citeonline{Martinhago2005} percebeu que uma das regras mais importante como a escolaridade dos pais, influenciavam na aprovação. Alguns dos fatores interessantes que foram extraídos é de que nos cursos menos concorridos os candidatos que foram aprovados, tinham concluído o Ensino Médio a mais de 5 anos e que vinham de escolas públicas ao contrário dos cursos mais concorridos onde essa relação não era frequente, outro fator, é de o candidato ter ou não feito cursinho, não influenciava no resultado.

\par
Para a base de dados contendo os dados de todos os candidatos ao vestibular, \citeonline{Martinhago2005} notou que para as regras geradas, a maioria dos candidatos moram com os pais, não trabalham e que estão em uma faixa de 17 a 20 anos. Observou também, que as pontuações obtidas pelos candidatos para um curso onde na área dele a nota de algumas disciplinas são essenciais para o vestibular, não tem tanta influência no resultado como a pontuação obtida em outras disciplinas de outras áreas, por exemplo na área de exatas, como o curso de Estatística onde as notas de Geografia, Língua estrangeira e História influenciam na aprovação do candidato, ao contrário da área de humanas como Ciências Sociais, onde notas como Matemática, Física e Química, influenciam nos resultados de aprovação. 

\par
Assim como os trabalhos anteriores que são semelhantes a este TCC proposto, o trabalho de \citeonline{Martinhago2005} usa como base de dados para o seus experimentos os dados dos questionários socioeconômico e os dados do cadastro geral dos candidatos do vestibular da UFPR para o seus testes de mineração, a diferença é que \citeonline{Martinhago2005} fez junção com alguns dados fornecidos pelo ENEM para produzir mais regras e ter um melhor resultado na sua classificação. Outro fator semelhante é que é utilizado a técnica de mineração a Árvore de Decisão usando o algoritmo J48 pela ferramenta WEKA para a geração de regras para se obter os resultados. Assim como \citeonline{Martinhago2005} encontrou alguns dados em branco e atributos repetidos e teve que adaptar para o software WEKA, presumo que neste trabalho os mesmos problemas durante a seleção e pré-processamento de dados serão encontrados.

%==========================================================
% REVISÃO SISTEMÁTICA
%==========================================================
\section{Síntese dos Trabalhos Correlatos}

\par
Foram analisados na literatura, trabalhos que discorrem sobre a utilização de mineração de dados em uma base de dados de um vestibular, para a criação de um modelo de classificação ou predição através de regras criadas, a fim de gerar perfis dos candidatos que são aprovados ou saber quais foram os fatores que afetaram ou contribuíram no desempenho do candidato na sua nota da prova.

\par
Em \citeonline{Simon2017}, motivados pelos problemas de infraestrutura apresentados pelo Censo Escolar da Educação Básica de 2016, criaram um modelo preditivo que relacionava o perfil socioeconômico dos candidatos fornecidos pelo INEP com a infraestrutura das escolas apresentados pelo PISA para associar ao desempenho dos alunos em exames, no caso, a prova do ENEM.

\par
\citeonline{Silva2014}, em suas pesquisas identificaram fatores que diminuem o desempenho dos candidatos, através da técnica de mineração baseada em regras de associação que utilizavam diferentes parametrizações do algoritmo apriori. Dados esses utilizados dos questionários socioeconômicos preenchidos pelos candidatos na edição de 2010 do ENEM da região Sudeste. Essas informações, especificamente as respostas de quatro perguntas, são relacionadas com o resultado de desempenho desses alunos.

\par
Por fim, \citeonline{Martinhago2005} investigou a base de dados sobre os vestibulandos da UFPR de 2003, como os dados dos questionários socioeconômico preenchidos pelos candidatos durante as inscrições, os dados do cadastro geral que contém as notas obtidas da prova e redação e o resultado do vestibular em conjunto com as notas e média das notas obtidas pelo ENEM, para poder traçar o perfil dos vestibulandos, utilizando a técnica de mineração árvore de decisão.

\par
A Tabela 1 visa prestar uma breve comparação entre as características chaves dos trabalhos relacionados e as propostas deste trabalho.

\begin{table}[h!]
\caption{Comparação entre os trabalhos correlacionados e a proposta deste trabalho.}
\centering
\begin{tabular}{c c c c c c c}
\hline \vspace{0.1cm}
      & PAB & EF & WE & TC & CT & TA \\ \hline 
    \vspace{0.1cm} \citeonline{Simon2017} & X & X & X &  &  & X \\ 
    \vspace{0.1cm} \citeonline{Silva2014} & X & X &   & X &  & X \\
    \vspace{0.1cm} \citeonline{Martinhago2005} & X &   & X & X &  &  \\
    \vspace{0.1cm} Este Trabalho & X & X & X & X & X & X \\ \hline 
\end{tabular}
\end{table}


\begin{itemize}
%\item \textbf{Q1:} Quais são os métodos para verificação de circuitos lógicos descritos na linguagem de programação VHDL?
	%\begin{itemize}
	\item \textbf{PAB:} Pré-processamento e análise do banco de dados.
	\item \textbf{EF:} Extração das \textit{features}.
	\item \textbf{WE:} Utilização da ferramenta WEKA para a extração de dados.
	\item \textbf{TC:} Utilização de técnicas de mineração de dados para gerar um modelo de classificação.
	\item \textbf{CT:} Comparativo entre as técnicas de mineração para saber qual foi a mais eficiente.
	\item \textbf{TA:} Utilização de técnicas para avaliar os resultados obtidos.
	%\end{itemize}
\end{itemize}






%1.Pré-processamento e analise da base de dados dos candidatos.
%2.Adotar um algoritmo de aprendizagem de máquina para identificar os fatores que influên-ciam o desempenho dos candidatos.
%3.Definir e aplicar uma técnica de associação de dados, para consolidar os dados dos fatoresde desempenho dos candidatos.
%4.  Definir métodos e técnicas para validar os resultados obtidos.




% ----------------------------------------------------------
% Detalhes de Desenvolvimento do Projeto
% ----------------------------------------------------------
\chapter{MÉTODO PROPOSTO}
\label{chapter:metodo}
Este Capítulo descreve o método proposto neste trabalho, baseado em VHDL para analise de circuitos utilizando transformações de código e a ferramenta ESBMC.

\section{Visão geral do método}
O método consiste na análise de código VHDL de um circuito lógico através da ferramenta ESBMC. O método é apresentado na \autoref{fig:bpmn}, onde consiste inicialmente em um circuito lógico descrito em VHDL,com as assertivas a serem verificadas, posteriormente traduzida para linguagem C através de uma ferramenta chamada V2C. Após a tradução inicial, o código é reescrito, de modo que as assertivas presente no código VHDL sejam aceitas pelo ESBMC. Baseado nas assertivas o ESBMC analisa o código, de modo a verificar, caso em algum momento, uma das assertivas seja violada, caso ocorra, a mesma apresenta a assertiva violada.

\begin{figure}[htb]
	\begin{center}
    \caption{\label{fig:bpmn}Ciclo da ferramenta}
	\includegraphics[scale=0.4]{Figuras/bpmn.png}
	\end{center}
    \legend{Fonte: Própria}
\end{figure}

\section{\label{cap:vhdl_assertivas}Código VHDL e assertivas}
\par
Na \autoref{fig:code_and} apresnta a estrutura geral do algoritmo na linguagem VHDL que será analisado pela ferramenta. O código não apresenta alteração na estrutura utilizada pela própria linguagem, composto de entidade e arquitetura. A utilização da linguagem é limitada apenas pela feramenta de tradução, V2C, que será apresentada na sessão \autoref{cap:traducao} e diferença no modo que assertiva será declarada, visto que a linguagem possui a utilização de assertivas nativamente.

\par
As assertivas, como citado anteriomente, apresenta estrutura diferente da utilizada pelo padrão do vhdl, as mesmas serão adicionadas ao código por meio das \textcolor{red}{tags} \textbf{@c2vhdl:ASSERT} e \textbf{@c2vhdl:END} e todo trecho de código entre estas \textcolor{red}{tags} deve esta comentado, desta forma não apresentará erro na execução do código em VHDL, caso o mesmo seja necessário. O trecho entre as \textcolor{red}{tags} apresentará trés informações necessárias:
\begin{enumerate}
\item \textbf{Condição:} A condição é a assertiva propriamente dita e que será analisado pela aplicação. A assertiva será precedida da palavra \textbf{assert} e/ou da palavra \textbf{not}, com isso a assertiva pode assumir propriedade negativa, conforme necessidade do usuário. A assertiva deve ser apresentada entre parenteses, conforme exemplificado na \autoref{fig:code_and}.
\item \textbf{Mensagem:} A mensagem é apresentada ao usuario, caso a assertiva falhe, pode ser indicando a propriedade violado, por exemplo ou quaquer outra mensagem a ser apresentada. E precedida pela palavra \textbf{report}.
\item \textbf{Gravidade:} O usuário pode um nível de gravidade, pode ser \textit{error} ou \textit{warning}. É precedido pela palavra \textbf{severity}.
\end{enumerate}

\begin{figure}[thp]
\caption{\label{fig:code_and} Exemplo de porta AND em VHDL com padrão da assertivas a ser adotada pelo método.}
	\begin{center}
    \begin{minipage}{0.6\textwidth}
    \begin{lstlisting}       
library ieee;
use ieee.std_logic_1164.all;

entity AND_ent is
port(   x, y: in bit;
        F: out bit
);
end AND_ent;

architecture behav1 of AND_ent is
begin
    --@c2vhdl:ASSERT
    --assert (x='__VERIFIER_nondet_int()' and y='__VERIFIER_nondet_int()')
    --report "Both values of signals x and y are equal to 1"
    --severity ERROR;
    --@c2vhdl:END
    
    process(x, y)
    begin
        if ((x='1') and (y='1')) then
            F <= '1';
        else
            F <= '0';
        end if;
    end process;
end behav1;

\end{lstlisting}
    \end{minipage}
	\end{center}
    \legend{Fonte: Própria.}
\end{figure}

\section{\label{cap:traducao}Tradução para código em linguagem C}
A tradução é utilizando a ferramenta V2c, e devido a ser uma ferramenta antiga a mesma apresenta certas limitação na tradução do código VHDL para ser, em outras palavras, a mesma não reconhece algumas estruturas especificas do VHDL. A ferramenta aceita apenas entradas e saídas do tipo: bit, std\underline{\space}ulogic, qsim\underline{\space}state, std\underline{\space}ulogic\underline{\space}vector e interger. Na parte de arquitetura, a ferramenta aceita uma gama maior de estruturas, operandos em expressoões do tipo: signal, variable, integers, strings e caracteres. Em expreções condicionais os operandos são AND, OR, NOT <=, =>,=,<,> e <>. Aceita também a estrutura de process, além da estutura block. A estrutura \textit{process} é limitada apenas a: if-else, case e  loops.

\par
Conforme especificado e seguindo os parametros da ferramenta, a mesma realiza a tradução, mantendo inalterado qualquer fragmento de código que esteja comentado, neste caso, qualquer assertiva existente no corpo do texto.
\section{Geraçao de código intermediário} 
\par
O código intrmediario é gerado a partir do código traduzido pelo V2C trabalhando diretamente nas assertivas comentadas no corpo do código. Devido as \textcolor{red}{tags} utilizadas e utilizando a tecnica de Regex, as assertivas são localizadas e traduzidas.

\par
A \autoref{fig:code_assert} apresenta um exemplo da assertiva em código VHDL, conforme explanado na sessão \autoref{cap:vhdl_assertivas}, a partir deste trecho a assertiva é inserida ao código para posteriomente ser analisada pelo ESBMC.

\begin{figure}[thp]
\caption{\label{fig:code_assert} Exemplo de assertiva seguindo o padrão proposto no método e apresentado na \autoref{fig:code_and}}. 
	\begin{center}
    \begin{minipage}{0.7\textwidth}
    \begin{lstlisting}       	
    --@c2vhdl:ASSERT
    --assert (x='__VERIFIER_nondet_int()' and y='__VERIFIER_nondet_int()')
    --report "Both values of signals x and y are equal to 1"
    --severity ERROR;
    --@c2vhdl:END
\end{lstlisting}
    \end{minipage}
	\end{center}
    \legend{Fonte: Própria.}
\end{figure}

\section{Analise ESBMC}



% ----------------------------------------------------------
% Cronograma
% ----------------------------------------------------------
\chapter{CRONOGRAMA}
\label{chapter:cronograma}

\par
Esse capítulo apresenta o conograma de desenvolvimento das atividades que serão realizadas no TCC I e II.

As atividades que serão desenvolvidas no TCC I e II, estão listas abaixo e representadas na Tabela 2:

\begin{enumerate}
  \item Pesquisar e estudar o referencial teórico.    
  \item Escrever o TCC I.
  \item Revisar e concluir o TCC I.
  \item Apresentar/Defender o TCC I.
  \item Pré-processamento dos dados forncidos pela CPV.
  \item Aplicar as técnicas de mineração no banco de dados.
  \item Identificar os fatores que influenciam o desempenho dos candidatos.
  \item Analisar os resultados alcançados. 
  \item Redigir o TCC II.
  \item Revisar e concluir o TCC II.
  \item Apresentar e defender o TCC II.
\end{enumerate}

\textcolor{red}{Apresentar tabela: 13 de agosto a 14 dezembro}.
\label{chapter:cronograma}
\begin{table}[htbp]
  \centering
  \caption{Cronograma de atividades}
  \label{tab:cronograma}
  \begin{tabularx}{\textwidth}{|X|c|c|c|c|c|}
    \hline
    \textbf{Atividade} & \textbf{Agosto} & \textbf{Setembro} & \textbf{Outubro} & \textbf{Novembro} & \textbf{Dezembro} \\
    \hline
    Revisão da literatura sobre verificação de hardware com assertivas & \(\times\) & \(\times\) & & & \\
    \hline
    Busca e teste de ferramenta para traução de código VHDL para C & \(\times\) & & & & \\
    \hline
    Definição e implementação de assertivas da execução automática da ferramenta & & \(\times\) & \(\times\) & & \\
    \hline
    Adicionar Bounded model checking no método & & \(\times\) & \(\times\) & &  \\
    \hline
    Finalização da implementação das funcionalidades da ferramenta & & & \(\times\) & \(\times\) &  \\
    \hline
    Realização de testes com benchmarks públicos escritos em VHDL & & &  & \(\times\) & \\
    \hline
    Finalização monografia & & & & \(\times\) & \(\times\) \\
    \hline
    Apresentação final & & & &  & \(\times\) \\
    \hline
  \end{tabularx}
\end{table}

% ----------------------------------------------------------
% Resultados -- Pode vir junto com discussão
% ----------------------------------------------------------
%\chapter{RESULTADOS EXPERIMENTAIS}
%\input{Sections/Resultados.tex}
%\chapter{DISCUSSÃO}

% ----------------------------------------------------------
% Conclusão
% ----------------------------------------------------------
\chapter{CONSIDERAÇÕES FINAIS}
\label{chapter:consideracoes}


\par 
A mineração de dados se tornou uma ferramenta de suporte com papel essencial no gerenciamento da informação dentro das organizações. O manuseamento dos dados e a análise das informações que eram feitas de maneira convencional, se tornou impossível devido a imensa quantidade de dados que é coletado e armazenado em sua base diariamente. Encontrar padrões escondidos e relacionamentos em arquivos que possuem um grande volume de informações de forma manual, deixou de ser uma escolha. As técnicas de mineração passaram a ser utilizadas e começaram a estar presentes no nosso cotidiano.

\par
Com uma grande quantidade de dados que estão sendo armazenados pelas universidades e com recente surgimento de várias tecnologias e técnicas de extração de dados, uma boa parte dessas instituições estão tentando de alguma forma utilizar esses dados que são extraídos  na intenção de compreende-los e utiliza-los para benefício próprio. Como podemos observar nos trabalhos apresentado neste TCC, que através dos dados armazenado de candidatos que vão prestar o vestibular, foi gerado um perfil dos mesmos na intenção de descobrir os motivos que influenciam os seus desempenhos na nota da prova que eles fizeram.

\par
Assim como os trabalhos apresentados, este trabalho possui o mesmo objetivo que os deles, que é a de gerar um perfil dos candidatos para saber os fatores que influenciam o seu desempenho na pontuação da prova, com intuito de que com os resultados obtidos, de alguma forma, possa ofereça como apoio para que o MEC analise a sua estratégia no plano de ensino das escolas e também para que o MDS com o conhecimento obtido desses resultados, possa solucionar os problemas socioeconômico dos candidatos que foram encontrado. Se espera também, que com o perfil dos candidatos que foram gerados, a UFRR tome futuras ações com essas informações para benefício próprio.

Em resumo, podemos perceber que a mineração de dados vem  cada vez mais sendo utilizada dentro da área educacional como em universidades, pois, com a imensa quantidade de dados que é armazenado, onde muitas vezes é repleta de informações importantes escondidas, se necessita dessas técnicas de mineração para se extrair esses dados com a intenção de usa-las, podendo assim relacionar informações contidas para predeterminar possíveis ações futuras na sua gestão com tutores, alunos e entre outros. Não resta dúvida de que a mineração de dados na área educacional está sendo extremamente promissora e que, apesar dos resultados já alcançados, ainda ela tem muito para o que oferecer.


%\newpage
% ----------------------------------------------------------
% Referências bibliográficas
% ----------------------------------------------------------
\bibliography{main}

%---------------------------------------------------------------------
% INDICE REMISSIVO
%---------------------------------------------------------------------
%\phantompart
%\printindex
%---------------------------------------------------------------------

\end{document}

